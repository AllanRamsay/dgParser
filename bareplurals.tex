
\documentclass[11pt]{article}
\evensidemargin=0in 
\oddsidemargin=0in
\textwidth=6.5in
\topmargin=-1.0in
\textheight=10.5in
\usepackage{defns}
\usepackage{examples}
\usepackage{pstricks}
\usepackage{fancyvrb}
\usepackage{natbib}
\bibliographystyle{elsart-harv}

\begin{document}
\noindent
I can't see much difference between

\begin{examples}
\item \label{S1}
\begin{examples}
\item
John was eating some peaches 
\item
John was eating peaches
\end{examples}
\end{examples}

\noindent
In both cases, eating was going on, there were peaches present, and
they were what was being eaten. So it's not an exciting example, but
we do need to preserve it.

I think that \xref{S2} is more interesting.

\begin{examples}
\item \label{S2}
\begin{examples}
\item
John eats some peaches 
\item
John eats peaches
\end{examples}
\end{examples}

\noindent
It seems to me that there is a clear difference between \xref{S2}(a) and
\xref{S2}(b). In \xref{S2}(a) there is an explicit set of peaches
which he eats (and by implicature there are some peaches that he
doesn't eat). In \xref{S2}(b) there is \textbf{no explicit set of
  peaches}.

But, with no further context I'd say that \xref{S2}(a) and
\xref{S2}(b) denote \textbf{sets} of eating events. They say that
there are (plural) sets of John-eating-peach events. I'm perfectly
comfortable with the idea that sentences report sets of events, with
the cardinality of the set denoted by the aspect, in the same way that
I'm perfectly comfotable with the notion that all NPs denote sets of
things, some of which happen to be singleton sets.

But that gives us a handle \xref{S2}. They both say that there are
sets of eating events: but \xref{S2}(a) says that there is a set of
peaches which are the objects of these events, \xref{S2}(b) says that
there is a set of peaches for each event. So it's a scope
issue. They're both existentials, but \q{some peaches} outscopes the
aspect and \q{peaches} is outscoped by it.

To get a handle on this, we make the aspect explicitly introduce a set
of events. That's easy enough -- we just add a universal quantifier
inside the scope of the existential which we used to get from the
aspect. So for 

\begin{examples}
\item
John sleeps.
\end{examples}

\noindent
we used to get (before \texttt{qff}, because I think it's easier to
see what's going on while we have real quantifiers) (note that
reference now outscope \texttt{claim}: that's the right thing to be
doing -- see the end of this document for a brief commentary on how
this works).

\begin{Verbatim}[commandchars=\\\{\}]
name(A::{[John:NP],A},
     claim(exists(B::{[tense(present)],B},
                  exists(C::{(simple,B),C},
                         [[sleep, {subject,A}], C]))))
\end{Verbatim}

\noindent
and I now get 

\begin{Verbatim}[commandchars=\\\{\}]
name(A::{[John:NP],A},
     claim(exists(B::{[tense(present)],B},
                  exists(C::{(simple,B),C},
                         \textcolor{blue}{forall(D::{(member,C),D},}
                                [[sleep, {subject,A}], D])))))
\end{Verbatim}

\noindent
(\texttt{\{(member,C),D\}} here means \texttt{D} is a member of
\texttt{C}: slightly odd way to write it, but I can do that if I want)

OK, now using this treatment of aspect I can get different
interpretations of \xref{S2}(a) and
\xref{S2}(b) that bring out the difference.

\begin{examples}
\item [S2]
\begin{examples}
\item
John eats some peaches.

\begin{Verbatim}[commandchars=\\\{\}]
name(A::{[John:NP],A},
     claim(exists(B::{[tense(present)],B},
                  exists(C::{(simple,B),C},
                         \textcolor{blue}{exists(D::{[peach>plural],D},}
                                forall(E::{(member,C),E},
                                       [[eat, {dobj,D}, {subject,A}],
                                        E]))))))
\end{Verbatim}

\item
John eats peaches.

\begin{Verbatim}[commandchars=\\\{\}]
name(A::{[John:NP],A},
     claim(exists(B::{[tense(present)],B},
                  exists(C::{(simple,B),C},
                         forall(D::{(member,C),D},
                                \textcolor{blue}{exists(E::{peach>plural,E},}
                                       [[eat, {dobj,E}, {subject,A}],
                                        D]))))))
\end{Verbatim}
\end{examples}
\end{examples}

\noindent
That does nearly everything I want. In \xref{S2}(a) there is a set of
peaches and a set of events in which John eats that set. Note that
since you can only eat a given peach once, that more or less forces
the set of events to be a singleton, which might be nice. In
\xref{S2}(b) there's a different set of peaches for each event in the
set, which is definitely what I want.

This does sensible things with 

\begin{examples}
\item 
\begin{examples}
\item
John does not eat peaches.

\begin{Verbatim}[commandchars=\\\{\}]
name(A::{[John:NP],A},
     claim(not(exists(B::{[tense(present)],B},
                      exists(C::{(simple,B),C},
                             forall(D::{(member,C),D},
                                    exists(E::{peach>plural,E},
                                           [[do,
                                             {dobj,E},
                                             {subject,A}],
                                            D])))))))
\end{Verbatim}
\item
John is not eating peaches. 

\begin{Verbatim}[commandchars=\\\{\}]
the(A::{tense(present),A},
    name(B::{[John:NP],B},
         claim(not(exists(C::{(prog,A),C},
                          forall(D::{(member,C),D},
                                 exists(E::{peach>plural,E},
                                        [[eat,
                                          {dobj,E},
                                          {subject,B}],
                                         D])))))))
\end{Verbatim}
\end{examples}
\end{examples}

\noindent
We still have to do something useful with the aspect markers, but
that's stuff we can look at as we go along.

\newpage
So that was good, but we haven't yet looked at the syllogistc
versions.

\begin{examples}
\item
All cats are idiots.

\begin{Verbatim}[commandchars=\\\{\}]
claim(forall(A::{[cat>plural],A},
             exists(B::{[tense(present)],B},
                    exists(C::{(simple,B),C},
                           forall(D::{(member,C),D},
                                  exists(E::{idiot>plural,E},
                                         [[be,
                                           {predication(xbar(v(-),n(+))),E},
                                           {subject,A}],
                                          D]))))))
\end{Verbatim}
\end{examples}

\noindent
Well it's very long and unreadable, but it's probaly not wrong (!). If
\texttt{A} is a set of cats then there's a set \texttt{E} of idiots
and right now there's a relation between them. A painfully complex
looking relation, but a relation nonetheless. A bit of catching things
during the preprocessing or postprocessing stages will make this look
better. 

\medpara
\hrule

\medpara
To get to here I've done various things.

\medpara
(i) I've made \q{not} do sensible things. Syntactically it takes a
sentence as argument and returns a sentence: that's not unreasonable,
but it's a bit complicated because I want to make it take the whole
tensed sentence, so that it dominates the auxiliaries in things like
\q{John is not eating peaches}, and that makes things a bit
complicated. A bit complicated, but do-able, so we do it.


\begin{Verbatim}[commandchars=\\\{\}]
[.,
 arg(claim,
     A,
     [not,
      arg(negComp,
          *(time(tense(present),
                 aspect(simple),
                 aux(+),
                 def(+),
                 finite(tensed))),
          [[be],
           arg(auxComp,
               *(time(tense(present),
                      aspect(prog),
                      aux(-),
                      def(B),
                      finite(participle))),
               [[eat>ing,
                 arg(dobj, *(existential=10), peach>plural)],
                arg(subject, *(name), [John:NP])])])])]
\end{Verbatim}
 
\noindent
(ii) The tree above has a bit that looks like \texttt{[not,
  arg(negComp, \ldots)]}. We have to deal with this in two places --
in \texttt{timeSequence}, where it's a bit of a hack, and in
\texttt{qlf}, where we just get rid of it because it was just
something we stuck in the parse tree.

\medpara
(iii) I've changed \texttt{assignScopeScore} so that scope can be
assigned locally, e.g. in the lexicon, and then just extracted when we
see it. To date I'm just using it for generics, which I'm specifying
as \texttt{*(existential=10)} -- as existentials with scope value 10
(so they end up with very local scope). We can do the same with other
things, e.g. if we decide to think of \q{any} as a very wide scope
universal we could do it by putting it into the lexicon with specifier
equal to \texttt{specifier(*(universal=0.2))}.

\begin{examples}
\item
John is not eating any peaches.

\begin{Verbatim}[commandchars=\\\{\}]
the(A::{tense(present),A},
    name(B::{[John:NP],B},
         claim(forall(C::{[peach>plural],C},
                      not(exists(D::{(prog,A),D},
                                 forall(E::{(member,D),E},
                                        [[eat,
                                          {dobj,C},
                                          {subject,B}],
                                         E])))))))
\end{Verbatim}
\end{examples}

\medpara
(iv) I'm allowing arbitrary items to have scope. That's a slightly
weird sounding notion, but it makes sense. So \texttt{not} is a scoped
operator, and so are \texttt{claim} and \texttt{query}. That lets
reference outscope the discourse operators, which is also the right
thing to do. 

\noindent
There's quite a bit here to talk about. I'm fairly happy with what's
here, but we'll need to work our way through it. Especially what we're
going to do with the syllogistic versions. See you on Thursday. 

\medpara
(I've stuck in lots and lots more printing so I can see what I'm
doing: we can take that out later)
\end{document}