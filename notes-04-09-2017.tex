
\documentclass[11pt,a4paper]{article}
\evensidemargin=0in 
\oddsidemargin=0in
\textwidth=6.5in
\topmargin=-1.0in
\textheight=10.5in
\usepackage{defns}
\usepackage{natbib}
\usepackage{ulem}
\usepackage{fancyvrb}
\usepackage{enumitem}
\usepackage{pstricks, pst-node, pst-tree}
\usepackage{examples}
\bibliographystyle{elsart-harv}
\usepackage{amsthm}
\newcommand{\IN}{$\in$}
\newcommand{\sub}{$\subseteq$}

\begin{document}

\section{From derivation trees to normalised dependency trees}

\medpara
(preamble: there are two terms which are \textbf{not} interchangeable. A
\defn{determiner} is a word with specific syntactic properties. A
\defn{specifier} is an item with a specific semantic function. Most
determiners function as specifiers most of the time, but they are not
the same notion
 (I may not be as careful about using these terms as I ought to be)).

\medpara
The grammar I am using is an HPSG/CG compromise. It's HPSG-like, but
there are only two rules of combination: heads combine with their arguments,
modifiers combine with their targets. The second of these two
conflates two things -- simple adjuncts, which copy a lot of features
from the target to the result, and specifiers, which change the value
of specified. \textbf{Some things do both}, which actually makes this
quite a nice rule. What I'm going to do is to say that actually
everything does both, but sometime the modifier bit is null, and
sometimes the specifier bit is null. So I have one rule for
everything, which is nice; but I have to do some tidying up afterwards
to remve null bits, which is less nice. Swings vs.\ roundabouts.

\medpara 
We start with a simple example:

\begin{examples}
\item He eats the ripe peaches.
\end{examples}

Doing this gives us \fig{he eats the ripe peaches .}

\begin{figure}[ht]
\centering{

\pstree[levelsep=80pt, nodesep=2pt,treesep=70pt, nrot=:D]{\TR{\begin{tabular}{|c|}\hline$.$\\\hline\end{tabular}}\nbput{}}{
 \pstree{\TR{\begin{tabular}{|c|}\hline$eat>s$\\\hline\*tense(present,-)\\\hline\end{tabular}}\nbput{claim}}{
  \pstree{\TR{\begin{tabular}{|c|}\hline$peach>s$\\\hline\*the\\\hline\end{tabular}}\nbput{dobj}}{
   \TR{\begin{tabular}{|c|}\hline$ripe>$\\\hline\end{tabular}}\nbput{amod}
   \TR{\begin{tabular}{|c|}\hline$the$\\\hline\end{tabular}}\nbput{}  }

  \TR{\begin{tabular}{|c|}\hline$he$\\\hline\*proRef\\\hline\end{tabular}}\nbput{subject} }
}
}
\caption{\q{he eats the ripe peaches .}}\label{he eats the ripe peaches .}
\end{figure}

There are two things to note about this tree before we proceed:

\begin{itemize}
\item
As noted above, some nodes carry a specifier, either because some lexical item
provided it (\q{the} in \q{the ripe peaches}) or because some
sublexical item provided it (\q{-s} in \q{eats}) or implicitly (\q{he}
is specified because it's a full NP by itself). Specifiers are marked
in the second line of the box.

\item
As also noted above, I don't really distinguish between specifiers and
modifiers, because some things can be both, and some can sometimes be
one and sometimes both. So I say that \textbf{every} specifier carries
the potential for including a modifier. However, some specifiers,
however, don't actually realise this potential. In such cases, the
label that \textbf{would have} marked this item as a modifier is
unlabelled. This leads to \fig{he eats the ripe peaches .'}.
\end{itemize}

\newpage
\begin{figure}[ht]
\centering{

\pstree[levelsep=70pt, nodesep=2pt,treesep=70pt, nrot=:D]{\TR{\begin{tabular}{|c|}\hline$.$\\\hline\end{tabular}}\nbput{}}{
 \pstree{\TR{\begin{tabular}{|c|}\hline$eat>s$\\\hline\*tense(present,-)\\\hline\end{tabular}}\nbput{claim}}{
  \pstree{\TR{\begin{tabular}{|c|}\hline$peach>s$\\\hline\*the\\\hline\end{tabular}}\nbput{dobj}}{
   \TR{\begin{tabular}{|c|}\hline$ripe>$\\\hline\end{tabular}}\nbput{amod}  }

  \TR{\begin{tabular}{|c|}\hline$he$\\\hline\*proRef\\\hline\end{tabular}}\nbput{subject} }
}
}
\caption{\q{he eats the ripe peaches .}, null modifier deleted}\label{he eats the ripe peaches .'}
\end{figure}

\noindent
The notion that trees may contain nodes that don't actually contribute
anything may be extended to other phenomena. The auxiliary in
\xref{WAS} contributes a specifier, but it doesn't add anything to the
propositional content. We therefore leave it without a label to its
daughter (\fig{he was eating the ripe peaches .}(a)), which means that it is not doing anything and hence can be
deleted (\fig{he was eating the ripe peaches .}(b)).

\begin{examples}
\item \label{WAS} he was eating the ripe peaches.
\end{examples}

\begin{figure}[ht]
\hspace*{\fill}
\begin{minipage}[t]{0.45\linewidth}
\pstree[levelsep=67pt, nodesep=2pt,treesep=50pt, nrot=:D]{\TR{\begin{tabular}{|c|}\hline$.$\\\hline\end{tabular}}\nbput{}}{
 \pstree{\TR{\begin{tabular}{|c|}\hline$be$\\\hline\*tense(past,+)\\\hline\end{tabular}}\nbput{claim}}{
  \pstree{\TR{\begin{tabular}{|c|}\hline$eat>ing$\\\hline\end{tabular}}\nbput{}}{
   \pstree{\TR{\begin{tabular}{|c|}\hline$peach>s$\\\hline\*the\\\hline\end{tabular}}\nbput{dobj}}{
    \TR{\begin{tabular}{|c|}\hline$ripe>$\\\hline\end{tabular}}\nbput{amod}   }

   \TR{\begin{tabular}{|c|}\hline$he$\\\hline\*proRef\\\hline\end{tabular}}\nbput{subject}  }
 }
}

\centering{(a)}
\end{minipage}
\hspace*{\fill}
\begin{minipage}[t]{0.45\linewidth}
\pstree[levelsep=67pt, nodesep=2pt,treesep=50pt, nrot=:D]{\TR{\begin{tabular}{|c|}\hline$.$\\\hline\end{tabular}}\nbput{}}{
 \pstree{\TR{\begin{tabular}{|c|}\hline$eat>ing$\\\hline\*tense(past,+)\\\hline\end{tabular}}\nbput{claim}}{
  \pstree{\TR{\begin{tabular}{|c|}\hline$peach>s$\\\hline\*the\\\hline\end{tabular}}\nbput{dobj}}{
   \TR{\begin{tabular}{|c|}\hline$ripe>$\\\hline\end{tabular}}\nbput{amod}  }

  \TR{\begin{tabular}{|c|}\hline$he$\\\hline\*proRef\\\hline\end{tabular}}\nbput{subject} }
}

\centering{(b)}
\end{minipage}

\hspace*{\fill}
\caption{\q{he was eating the ripe peaches .}}\label{he was eating the ripe peaches .}
\end{figure}


We will \textbf{not} do this with modals. Modals have similar syntactic
properties to auxiliaries, and they also contribute a specifier, but
they do make a contribution to the propositional
content. (\ref{MIGHT-CAN}a) and  (\ref{MIGHT-CAN}b) do not mean the
same thing.

\begin{examples}
\item \label{MIGHT-CAN}
\begin{examples}
\item He might eat the unripe peaches .
\item He can eat the unripe peaches .
\end{examples}
\end{examples}


\begin{figure}[ht]
\hspace*{\fill}
\begin{minipage}[t]{0.45\linewidth}

\pstree[levelsep=80pt, nodesep=2pt,treesep=70pt, nrot=:D]{\TR{\begin{tabular}{|c|}\hline$.$\\\hline\end{tabular}}\nbput{}}{
 \pstree{\TR{\begin{tabular}{|c|}\hline$might$\\\hline\*tense(present,+)\\\hline\end{tabular}}\nbput{claim}}{
  \pstree{\TR{\begin{tabular}{|c|}\hline$eat>$\\\hline\end{tabular}}\nbput{modalcomp}}{
   \pstree{\TR{\begin{tabular}{|c|}\hline$peach>s$\\\hline\*the\\\hline\end{tabular}}\nbput{dobj}}{
    \TR{\begin{tabular}{|c|}\hline$ripe>$\\\hline\end{tabular}}\nbput{amod}   }

   \TR{\begin{tabular}{|c|}\hline$he$\\\hline\*proRef\\\hline\end{tabular}}\nbput{subject}  }
 }
}
\end{minipage}
\hspace*{\fill}
\begin{minipage}[t]{0.45\linewidth}

\pstree[levelsep=80pt, nodesep=2pt,treesep=70pt, nrot=:D]{\TR{\begin{tabular}{|c|}\hline$.$\\\hline\end{tabular}}\nbput{}}{
 \pstree{\TR{\begin{tabular}{|c|}\hline$can$\\\hline\*tense(present,+)\\\hline\end{tabular}}\nbput{claim}}{
  \pstree{\TR{\begin{tabular}{|c|}\hline$eat>$\\\hline\end{tabular}}\nbput{modalcomp}}{
   \pstree{\TR{\begin{tabular}{|c|}\hline$peach>s$\\\hline\*the\\\hline\end{tabular}}\nbput{dobj}}{
    \TR{\begin{tabular}{|c|}\hline$ripe>$\\\hline\end{tabular}}\nbput{amod}   }

   \TR{\begin{tabular}{|c|}\hline$he$\\\hline\*proRef\\\hline\end{tabular}}\nbput{subject}  }
 }
}
\end{minipage}
\hspace*{\fill}
\caption{\q{he might/can eat the ripe peaches .}}\label{he might/can eat the ripe peaches .}
\end{figure}

We now return to determiners. The first complication arises when we
look at numbers. We start with a simple case:

\begin{examples}
\item He ate six ripe peaches.
\end{examples}

\begin{figure}[ht]
\hspace*{\fill}
\begin{minipage}[t]{0.45\linewidth}

\pstree[levelsep=80pt, nodesep=2pt,treesep=70pt, nrot=:D]{\TR{\begin{tabular}{|c|}\hline$.$\\\hline\end{tabular}}\nbput{}}{
 \pstree{\TR{\begin{tabular}{|c|}\hline$ate$\\\hline\*tense(past,-)\\\hline\end{tabular}}\nbput{claim}}{
  \pstree{\TR{\begin{tabular}{|c|}\hline$peach>s$\\\hline\*indefinite\\\hline\end{tabular}}\nbput{dobj}}{
   \TR{\begin{tabular}{|c|}\hline$ripe>$\\\hline\end{tabular}}\nbput{amod}
   \TR{\begin{tabular}{|c|}\hline$six$\\\hline\end{tabular}}\nbput{numAsMod}  }

  \TR{\begin{tabular}{|c|}\hline$he$\\\hline\*proRef\\\hline\end{tabular}}\nbput{subject} }
}
\end{minipage}
\hspace*{\fill}
\caption{\q{he ate six ripe peaches .}}\label{he ate six ripe peaches .}
\end{figure}

\noindent
\q{six} is a modifier which specifies the cardinality of the set of
peaches, and also a specifier which marks the whole NP as
indefinite. The outcome seems reasonable enough, even if the way I'm
getting there is a bit unorthodox, since I'm treating \q{six} as a
modifier \textbf{and} a specifier (\fig{he ate six ripe peaches .}).

\newpage
But we can combine other determiners with \q{six}: 

\begin{examples}
\item
\begin{examples}
\item he ate the six peaches.
\item he ate some six peaches.\footnote{This one is slightly
mannered, but definitely sayable: \q{Petipa studied and taught fur some six years in Madrid before returning},
\texttt{BNC/A/A1/A12}: around 400 examples in the BNC, often but not
overwhelmingly with
temporal PPs as here.}
\end{examples}
\end{examples}

\begin{figure}[ht]
\hspace*{\fill}
\begin{minipage}[t]{0.45\linewidth}

\pstree[levelsep=80pt, nodesep=2pt,treesep=70pt, nrot=:D]{\TR{\begin{tabular}{|c|}\hline$.$\\\hline\end{tabular}}\nbput{}}{
 \pstree{\TR{\begin{tabular}{|c|}\hline$ate$\\\hline\*tense(past,-)\\\hline\end{tabular}}\nbput{claim}}{
  \pstree{\TR{\begin{tabular}{|c|}\hline$peach>s$\\\hline\*the\\\hline\end{tabular}}\nbput{dobj}}{
   \TR{\begin{tabular}{|c|}\hline$ripe>$\\\hline\end{tabular}}\nbput{amod}
   \TR{\begin{tabular}{|c|}\hline$six$\\\hline\end{tabular}}\nbput{numAsMod}  }

  \TR{\begin{tabular}{|c|}\hline$he$\\\hline\*proRef\\\hline\end{tabular}}\nbput{subject} }
}
\end{minipage}
\hspace*{\fill}\begin{minipage}[t]{0.45\linewidth}

\pstree[levelsep=80pt, nodesep=2pt,treesep=70pt, nrot=:D]{\TR{\begin{tabular}{|c|}\hline$.$\\\hline\end{tabular}}\nbput{}}{
 \pstree{\TR{\begin{tabular}{|c|}\hline$ate$\\\hline\*tense(past,-)\\\hline\end{tabular}}\nbput{claim}}{
  \pstree{\TR{\begin{tabular}{|c|}\hline$peach>s$\\\hline\*indefinite\\\hline\end{tabular}}\nbput{dobj}}{
   \TR{\begin{tabular}{|c|}\hline$ripe>$\\\hline\end{tabular}}\nbput{amod}
   \TR{\begin{tabular}{|c|}\hline$six$\\\hline\end{tabular}}\nbput{numAsMod}  }

  \TR{\begin{tabular}{|c|}\hline$he$\\\hline\*proRef\\\hline\end{tabular}}\nbput{subject} }
}
\end{minipage}
\hspace*{\fill}
\caption{\q{he ate the/some six ripe peaches .}}\label{he ate the/some six ripe peaches .}
\end{figure}

This time \q{six} is just a modifier, fixing the cardinality of the
set of peaches, and \q{the}/\q{some} are the specifiers. \q{six} isn't
a specifier because \q{six ripe peaches} isn't \texttt{+specified}:
a subtree only get to be specified if the external context demands it,
and in the current examples this is not the case.

\newpage
We also have to allow NPs with no determiner:

\begin{examples}
\item 
\begin{examples}
\item
He was eating ripe peaches .
\item
He does not like unripe ones.
\end{examples}
\end{examples}
\begin{figure}[ht]
\hspace*{\fill}
\begin{minipage}[t]{0.45\linewidth}

\pstree[levelsep=80pt, nodesep=2pt,treesep=70pt, nrot=:D]{\TR{\begin{tabular}{|c|}\hline$.$\\\hline\end{tabular}}\nbput{}}{
 \pstree{\TR{\begin{tabular}{|c|}\hline$eat>ing$\\\hline\*tense(past,+)\\\hline\end{tabular}}\nbput{claim}}{
  \pstree{\TR{\begin{tabular}{|c|}\hline$peach>s$\\\hline\*generic\\\hline\end{tabular}}\nbput{dobj}}{
   \TR{\begin{tabular}{|c|}\hline$ripe>$\\\hline\end{tabular}}\nbput{amod}  }

  \TR{\begin{tabular}{|c|}\hline$he$\\\hline\*proRef\\\hline\end{tabular}}\nbput{subject} }
}
\end{minipage}
\hspace*{\fill}
\begin{minipage}[t]{0.45\linewidth}

\pstree[levelsep=80pt, nodesep=2pt,treesep=70pt, nrot=:D]{\TR{\begin{tabular}{|c|}\hline$.$\\\hline\end{tabular}}\nbput{}}{
 \pstree{\TR{\begin{tabular}{|c|}\hline$like>$\\\hline\*tense(present,+)\\\hline\end{tabular}}\nbput{claim}}{
  \pstree{\TR{\begin{tabular}{|c|}\hline$one>s$\\\hline\*generic\\\hline\end{tabular}}\nbput{dobj}}{
   \TR{\begin{tabular}{|c|}\hline$unripe>$\\\hline\end{tabular}}\nbput{amod}  }

  \TR{\begin{tabular}{|c|}\hline$he$\\\hline\*proRef\\\hline\end{tabular}}\nbput{subject} }
}
\end{minipage}
\hspace*{\fill}
\caption{\q{he does not like unripe ones .}}\label{he does not like unripe ones .}
\end{figure}

\medpara
I have not yet seen a treatment of bare plurals that I like. My
preferred interpretation remains the one that I gave in
\citep{Ramsay:92b}. For now I will just introduce a specifier called
\texttt{generic}, and work out what it means later.

And then we get determiners that don't seem to need a noun to specify:

\begin{examples}
\item 
\begin{examples}
\item He ate some.
\item He did not eat any.
\end{examples}
\end{examples}

\begin{figure}[ht]
\hbox{
\pstree[levelsep=43pt, nodesep=2pt,treesep=70pt, nrot=:D]{\TR{$.$}\nbput{\scriptsize }}{
  \pstree{\TR{$*tense(past,-)$}\nbput{\scriptsize claim}}{
    \pstree{\TR{$ate$}\nbput{\scriptsize specifiee}}{
      \pstree{\TR{$some$}\nbput{\scriptsize dobj}}{
        \TR{$zero$}\nbput{\scriptsize specifiee}
      }
      \pstree{\TR{$[proRef]$}\nbput{\scriptsize subject}}{
        \TR{$he$}\nbput{\scriptsize specifiee}
      }
    }
  }
}
\hspace{0.5in}
\pstree[levelsep=43pt, nodesep=2pt,treesep=70pt, nrot=:D]{\TR{$.$}\nbput{\scriptsize }}{
  \pstree{\TR{$*tense(past,+)$}\nbput{\scriptsize claim}}{
    \pstree{\TR{$did$}\nbput{\scriptsize specifiee}}{
      \pstree{\TR{$eat>$}\nbput{\scriptsize auxcomp}}{
        \TR{$not$}\nbput{\scriptsize negation}
        \pstree{\TR{$any$}\nbput{\scriptsize dobj}}{
          \TR{$zero$}\nbput{\scriptsize specifiee}
        }
      }
      \pstree{\TR{$[proRef]$}\nbput{\scriptsize subject}}{
        \TR{$he$}\nbput{\scriptsize specifiee}
      }
    }
  }
}
}
\caption{he ate some/he did not eat any .}\label{he ate some/he did not eat any .}
\end{figure}

This is a fairly common phenomenon -- lots of determiners can take an
elliptical head noun. Even more striking, they can take an elliptical
head noun that has a post-modifier. Examples from the BNC -- (a)-(c)
have postmodifying PPs, (d)-(f) have postmodifying relative clauses:

\begin{examples}
\item\label{ZERONOUNS}
\begin{examples}
\item its peoples are very different from the Utopian fantasies \uline{of many in the developed world} who profess a passionate concern for Amazonia 
\item there are \uline{many in the world} who still recognize it as their own
\item \uline{some in Germany} may be tempted to seek reunification on
  the basis
\item not without some opposition from \uline{some who thought they
    were losing several days of their life}
\item there can not have been \uline{many who doubted West Indies
    would win}
\item they are vulnerable to the ordinary wear and tear of \uline{any who seek to serve among the wayward and weak}
\end{examples}
\end{examples}

We just allow this, at least for some determiners \fig{he was eating  some which she had given him .}
shows the analyses for a pair of sentences where one has \q{which she
  had given him} as a relative clause modifying \q{peaches} and one
has it as a relative clause with no target noun.

\begin{examples}
\item \label{RCLAUSES}
\begin{examples}
\item He was eating some peaches which she had given him. 
\item He was eating some which she had given him.
\end{examples}
\end{examples}

\begin{figure}[ht]
\hspace*{\fill}
\begin{minipage}[t]{0.45\linewidth}

\pstree[levelsep=80pt, nodesep=2pt,treesep=40pt, nrot=:D]{\TR{\begin{tabular}{|c|}\hline$.$\\\hline\end{tabular}}\nbput{}}{
 \pstree{\TR{\begin{tabular}{|c|}\hline$eat>ing$\\\hline\*tense(past,+)\\\hline\end{tabular}}\nbput{claim}}{
  \pstree{\TR{\begin{tabular}{|c|}\hline$peach>s$\\\hline\*indefinite\\\hline\end{tabular}}\nbput{dobj}}{
   \pstree{\TR{\begin{tabular}{|c|}\hline$given$\\\hline\*tense(past,+)\\\hline\end{tabular}}\nbput{whmod}}{
    \TR{\begin{tabular}{|c|}\hline$which$\\\hline\*whPron\\\hline\end{tabular}}\nbput{iobj}
    \TR{\begin{tabular}{|c|}\hline$him$\\\hline\*proRef\\\hline\end{tabular}}\nbput{obj}
    \TR{\begin{tabular}{|c|}\hline$she$\\\hline\*proRef\\\hline\end{tabular}}\nbput{subject}   }
  }

  \TR{\begin{tabular}{|c|}\hline$he$\\\hline\*proRef\\\hline\end{tabular}}\nbput{subject} }
}

\centering{a}
\end{minipage}\hspace*{\fill}
\begin{minipage}[t]{0.45\linewidth}

\pstree[levelsep=80pt, nodesep=2pt,treesep=40pt, nrot=:D]{\TR{\begin{tabular}{|c|}\hline$.$\\\hline\end{tabular}}\nbput{}}{
 \pstree{\TR{\begin{tabular}{|c|}\hline$eat>ing$\\\hline\*tense(past,+)\\\hline\end{tabular}}\nbput{claim}}{
  \pstree{\TR{\begin{tabular}{|c|}\hline$zero$\\\hline\*indefinite\\\hline\end{tabular}}\nbput{dobj}}{
   \pstree{\TR{\begin{tabular}{|c|}\hline$given$\\\hline\*tense(past,+)\\\hline\end{tabular}}\nbput{whmod}}{
    \TR{\begin{tabular}{|c|}\hline$which$\\\hline\*whPron\\\hline\end{tabular}}\nbput{iobj}
    \TR{\begin{tabular}{|c|}\hline$him$\\\hline\*proRef\\\hline\end{tabular}}\nbput{obj}
    \TR{\begin{tabular}{|c|}\hline$she$\\\hline\*proRef\\\hline\end{tabular}}\nbput{subject}   }
  }

  \TR{\begin{tabular}{|c|}\hline$he$\\\hline\*proRef\\\hline\end{tabular}}\nbput{subject} }
}

\centering{b}
\end{minipage}
\hspace*{\fill}
\caption{\q{he was eating some which she had given him .}}\label{he was eating some which she had given him .}
\end{figure}

\noindent
\q{some} should be a specifier: in \xref{RCLAUSES}(b) there is no noun for it to specify. \q{which she had given him} should be
a relative clause: in \xref{RCLAUSES}(b) there is no noun for it to modify. \uline{There is
  no noun at the heart of the NP \q{some which she had given him}}. In
such cases, the noun is elliptical and would normally have be to
determined with referece to the context -- in the context of the
examples so far, for instance, the zero noun in \xref{RCLAUSES}(b) is
almost certainly \q{peaches}!

\newpage
The examples in \xref{ZERONOUNS} and \xref{RCLAUSES} show that we have
to, and can, allow
for NPs with a determiner and a PP but no head noun. In particular, there are a big pile of determiners that can take \q{of
  the \ldots} (well actually \q{of NP[+def]}). Since we have had to
allow for PPs with zero heads to in order to cope with
\xref{ZERONOUNS}(a--f), we will simply use the same mechanism to
allow for \xref{OFZERO}. 

\begin{examples}
\item \label{OFZERO}
\begin{examples}
\item
He ate some of the ripe peaches. 
\item
He ate some of them.
\end{examples}
\end{examples}

\begin{figure}[ht]
\hspace*{\fill}
\begin{minipage}[t]{0.45\linewidth}

\pstree[levelsep=80pt, nodesep=2pt,treesep=70pt, nrot=:D]{\TR{\begin{tabular}{|c|}\hline$.$\\\hline\end{tabular}}\nbput{}}{
 \pstree{\TR{\begin{tabular}{|c|}\hline$eat>ing$\\\hline\*tense(past,+)\\\hline\end{tabular}}\nbput{claim}}{
  \pstree{\TR{\begin{tabular}{|c|}\hline$zero$\\\hline\*indefinite\\\hline\end{tabular}}\nbput{dobj}}{
   \pstree{\TR{\begin{tabular}{|c|}\hline$of$\\\hline\end{tabular}}\nbput{ppmod}}{
    \pstree{\TR{\begin{tabular}{|c|}\hline$peach>s$\\\hline\*the\\\hline\end{tabular}}\nbput{comp}}{
     \TR{\begin{tabular}{|c|}\hline$ripe>$\\\hline\end{tabular}}\nbput{amod}    }
   }
  }

  \TR{\begin{tabular}{|c|}\hline$he$\\\hline\*proRef\\\hline\end{tabular}}\nbput{subject} }
}
\end{minipage}
\hspace*{\fill}
\begin{minipage}[t]{0.45\linewidth}

\pstree[levelsep=80pt, nodesep=2pt,treesep=70pt, nrot=:D]{\TR{\begin{tabular}{|c|}\hline$.$\\\hline\end{tabular}}\nbput{}}{
 \pstree{\TR{\begin{tabular}{|c|}\hline$eat>ing$\\\hline\*tense(past,+)\\\hline\end{tabular}}\nbput{claim}}{
  \pstree{\TR{\begin{tabular}{|c|}\hline$zero$\\\hline\*indefinite\\\hline\end{tabular}}\nbput{dobj}}{
   \pstree{\TR{\begin{tabular}{|c|}\hline$of$\\\hline\end{tabular}}\nbput{ppmod}}{
    \TR{\begin{tabular}{|c|}\hline$them$\\\hline\*proRef\\\hline\end{tabular}}\nbput{comp}   }
  }

  \TR{\begin{tabular}{|c|}\hline$he$\\\hline\*proRef\\\hline\end{tabular}}\nbput{subject} }
}
\end{minipage}
\hspace*{\fill}
\caption{\q{he was eating some of them .}}\label{he was eating some of them .}
\end{figure}

\noindent
As before, we have a zero noun, but in these cases this noun is some
word like \q{group} or \q{set}:

\begin{examples}
\item 
\begin{examples}
\item He ate a bunch of the peaches.
\item He ate some of the peaches.
\end{examples}
\end{examples}

\newpage
\begin{figure}[ht]
\hspace*{\fill}
\begin{minipage}[t]{0.45\linewidth}

\pstree[levelsep=60pt, nodesep=2pt,treesep=70pt, nrot=:D]{\TR{\begin{tabular}{|c|}\hline$.$\\\hline\end{tabular}}\nbput{}}{
 \pstree{\TR{\begin{tabular}{|c|}\hline$ate$\\\hline\*tense(past,-)\\\hline\end{tabular}}\nbput{claim}}{
  \pstree{\TR{\begin{tabular}{|c|}\hline$bunch>$\\\hline\*indefinite\\\hline\end{tabular}}\nbput{dobj}}{
   \pstree{\TR{\begin{tabular}{|c|}\hline$of$\\\hline\end{tabular}}\nbput{ppmod}}{
    \TR{\begin{tabular}{|c|}\hline$[peach>s]$\\\hline\*the\\\hline\end{tabular}}\nbput{comp}   }
  }

  \TR{\begin{tabular}{|c|}\hline$he$\\\hline\*proRef\\\hline\end{tabular}}\nbput{subject} }
}
\end{minipage}
\hspace*{\fill}
\begin{minipage}[t]{0.45\linewidth}

\pstree[levelsep=60pt, nodesep=2pt,treesep=70pt, nrot=:D]{\TR{\begin{tabular}{|c|}\hline$.$\\\hline\end{tabular}}\nbput{}}{
 \pstree{\TR{\begin{tabular}{|c|}\hline$ate$\\\hline\*tense(past,-)\\\hline\end{tabular}}\nbput{claim}}{
  \pstree{\TR{\begin{tabular}{|c|}\hline$zero$\\\hline\*indefinite\\\hline\end{tabular}}\nbput{dobj}}{
   \pstree{\TR{\begin{tabular}{|c|}\hline$of$\\\hline\end{tabular}}\nbput{ppmod}}{
    \TR{\begin{tabular}{|c|}\hline$[peach>s]$\\\hline\*the\\\hline\end{tabular}}\nbput{comp}   }
  }

  \TR{\begin{tabular}{|c|}\hline$he$\\\hline\*proRef\\\hline\end{tabular}}\nbput{subject} }
}
\end{minipage}
\hspace*{\fill}
\caption{\q{he ate some of the peaches .}}\label{he ate some of the peaches .}
\end{figure}

\q{all} introduces a complication for which I do not have a
solution. At first sight, it seems to behave like \q{some} --
\fig{FIG:SOME/ALL} assigns exactly parallel analyses to \xref{SOME/ALL}:

\begin{examples}
\item \label{SOME/ALL}
\begin{examples}
\item
He ate some of the peaches.
\item
He ate all of the peaches.
\end{examples}
\end{examples}

\begin{figure}[ht]
\hspace*{\fill}
\begin{minipage}[t]{0.45\linewidth}

\pstree[levelsep=60pt, nodesep=2pt,treesep=70pt, nrot=:D]{\TR{\begin{tabular}{|c|}\hline$.$\\\hline\end{tabular}}\nbput{}}{
 \pstree{\TR{\begin{tabular}{|c|}\hline$ate$\\\hline\*tense(past,-)\\\hline\end{tabular}}\nbput{claim}}{
  \pstree{\TR{\begin{tabular}{|c|}\hline$zero$\\\hline\*indefinite\\\hline\end{tabular}}\nbput{dobj}}{
   \pstree{\TR{\begin{tabular}{|c|}\hline$of$\\\hline\end{tabular}}\nbput{ppmod}}{
    \TR{\begin{tabular}{|c|}\hline$[peach>s]$\\\hline\*the\\\hline\end{tabular}}\nbput{comp}   }
  }

  \TR{\begin{tabular}{|c|}\hline$he$\\\hline\*proRef\\\hline\end{tabular}}\nbput{subject} }
}
\end{minipage}
\hspace*{\fill}
\begin{minipage}[t]{0.45\linewidth}

\pstree[levelsep=60pt, nodesep=2pt,treesep=70pt, nrot=:D]{\TR{\begin{tabular}{|c|}\hline$.$\\\hline\end{tabular}}\nbput{}}{
 \pstree{\TR{\begin{tabular}{|c|}\hline$ate$\\\hline\*tense(past,-)\\\hline\end{tabular}}\nbput{claim}}{
  \pstree{\TR{\begin{tabular}{|c|}\hline$zero$\\\hline\*universal\\\hline\end{tabular}}\nbput{dobj}}{
   \pstree{\TR{\begin{tabular}{|c|}\hline$of$\\\hline\end{tabular}}\nbput{ppmod}}{
    \TR{\begin{tabular}{|c|}\hline$[peach>s]$\\\hline\*the\\\hline\end{tabular}}\nbput{comp}   }
  }

  \TR{\begin{tabular}{|c|}\hline$he$\\\hline\*proRef\\\hline\end{tabular}}\nbput{subject} }
}
\end{minipage}
\hspace*{\fill}
\caption{\q{he ate all of the peaches .}}\label{FIG:SOME/ALL}
\end{figure}

But \q{all} takes part in constructions where (i) the preposition
\q{of} can be omitted and (ii) \q{all} can float to a position after
the definite NP (Appendix \ref{APP:ALL} contains examples of these
constructions from the BNC). All the examples in \xref{ALL} seem to mean the same
thing, and it is very tempting to think that they all have the same
underlying structure. I can't make that happen: I can give
\xref{ALL}(b) and \xref{ALL}(c) the same structure, but I can't make
that the same as the structure of \xref{ALL}(b). the trees in
\fig{FIG:ALL} are the raw parse trees \textbf{before} we remove items
with no function: the big problem that I have (might seem like a small
one, but it isn't), is that I cannot retain the specifier
\texttt{*the} in the trees for \xref{ALL}(b) and \xref{ALL}(c) . We will return to
this: right now it's a puzzle.

\begin{examples}
\item \label{ALL}
\begin{examples}
\item
All of the men were eating peaches.
\item
All the men were eating peaches.
\item
The men were all eating peaches.
\end{examples}
\end{examples}

\begin{figure}[ht]
\hbox{\hspace*{\fill}
\begin{minipage}[t]{0.33\linewidth}
\pstree[levelsep=80pt, nodesep=2pt,treesep=30pt, nrot=:D]{\TR{\begin{tabular}{|c|}\hline$.$\\\hline\end{tabular}}\nbput{}}{
 \pstree{\TR{\begin{tabular}{|c|}\hline$be$\\\hline\*tense(past,+)\\\hline\end{tabular}}\nbput{claim}}{
  \pstree{\TR{\begin{tabular}{|c|}\hline$eat>ing$\\\hline\*tense(past,+)\\\hline\end{tabular}}\nbput{identity}}{
   \TR{\begin{tabular}{|c|}\hline$peach>s$\\\hline\*generic\\\hline\end{tabular}}\nbput{dobj}
   \pstree{\TR{\begin{tabular}{|c|}\hline$zero$\\\hline\*universal\\\hline\end{tabular}}\nbput{subject}}{
    \pstree{\TR{\begin{tabular}{|c|}\hline$of$\\\hline\end{tabular}}\nbput{ppmod}}{
     \pstree{\TR{\begin{tabular}{|c|}\hline$man>s$\\\hline\*the\\\hline\end{tabular}}\nbput{comp}}{
      \TR{\begin{tabular}{|c|}\hline$the$\\\hline\end{tabular}}\nbput{identity}     }
    }
    \TR{\begin{tabular}{|c|}\hline$all$\\\hline\end{tabular}}\nbput{identity}   }
  }
 }
}

\vspace{0.2in}
\centering{\xref{ALL}(a)}
\end{minipage}
\hspace*{\fill}
\begin{minipage}[t]{0.33\linewidth}
\pstree[levelsep=80pt, nodesep=2pt,treesep=30pt, nrot=:D]{\TR{\begin{tabular}{|c|}\hline$.$\\\hline\end{tabular}}\nbput{}}{
 \pstree{\TR{\begin{tabular}{|c|}\hline$be$\\\hline\*tense(past,+)\\\hline\end{tabular}}\nbput{claim}}{
  \pstree{\TR{\begin{tabular}{|c|}\hline$eat>ing$\\\hline\*tense(past,+)\\\hline\end{tabular}}\nbput{identity}}{
   \TR{\begin{tabular}{|c|}\hline$peach>s$\\\hline\*generic\\\hline\end{tabular}}\nbput{dobj}
   \pstree{\TR{\begin{tabular}{|c|}\hline$man>s$\\\hline\*universal\\\hline\end{tabular}}\nbput{subject}}{
    \TR{\begin{tabular}{|c|}\hline$the$\\\hline\end{tabular}}\nbput{identity}
    \TR{\begin{tabular}{|c|}\hline$all$\\\hline\end{tabular}}\nbput{}   }
  }
 }
}

\vspace{0.2in}
\centering{\xref{ALL}(b)}
\end{minipage}
\hspace*{\fill}
\begin{minipage}[t]{0.33\linewidth}
\pstree[levelsep=80pt, nodesep=2pt,treesep=30pt, nrot=:D]{\TR{\begin{tabular}{|c|}\hline$.$\\\hline\end{tabular}}\nbput{}}{
 \pstree{\TR{\begin{tabular}{|c|}\hline$be$\\\hline\*tense(past,+)\\\hline\end{tabular}}\nbput{claim}}{
  \pstree{\TR{\begin{tabular}{|c|}\hline$eat>ing$\\\hline\*tense(past,+)\\\hline\end{tabular}}\nbput{identity}}{
   \TR{\begin{tabular}{|c|}\hline$peach>s$\\\hline\*generic\\\hline\end{tabular}}\nbput{dobj}
   \pstree{\TR{\begin{tabular}{|c|}\hline$man>s$\\\hline\*universal\\\hline\end{tabular}}\nbput{subject}}{
    \TR{\begin{tabular}{|c|}\hline$the$\\\hline\end{tabular}}\nbput{identity}
    \TR{\begin{tabular}{|c|}\hline$all$\\\hline\end{tabular}}\nbput{}   }
  }
 }
}

\vspace{0.2in}
\centering{\xref{ALL}(c)}
\end{minipage}
\hspace*{\fill}
}
\caption{\q{he ate all of the peaches .}}\label{FIG:ALL}
\end{figure}

\newpage

We can now deal with a reasonably wide range of structures involving
determiners, including  ones with no determiner, ones with no noun,
and ones with complex sequences of determiners such as \xref{ALLSIXOFTHE}:

\begin{examples}
\item \label{ALLSIXOFTHE} he ate all six of the peaches .
\end{examples}

\begin{figure}[ht]
\hspace*{\fill}
\begin{minipage}[t]{0.45\linewidth}

\pstree[levelsep=80pt, nodesep=2pt,treesep=70pt, nrot=:D]{\TR{\begin{tabular}{|c|}\hline$.$\\\hline\end{tabular}}\nbput{}}{
 \pstree{\TR{\begin{tabular}{|c|}\hline$ate$\\\hline\*tense(past,-)\\\hline\end{tabular}}\nbput{claim}}{
  \pstree{\TR{\begin{tabular}{|c|}\hline$zero$\\\hline\*universal\\\hline\end{tabular}}\nbput{dobj}}{
   \pstree{\TR{\begin{tabular}{|c|}\hline$of$\\\hline\end{tabular}}\nbput{ppmod}}{
    \TR{\begin{tabular}{|c|}\hline$[peach>s]$\\\hline\*the\\\hline\end{tabular}}\nbput{comp}   }

   \TR{\begin{tabular}{|c|}\hline$six$\\\hline\end{tabular}}\nbput{numAsMod}  }

  \TR{\begin{tabular}{|c|}\hline$he$\\\hline\*proRef\\\hline\end{tabular}}\nbput{subject} }
}
\end{minipage}
\hspace*{\fill}
\caption{\q{he ate all six of the peaches .}}\label{he ate all six of the peaches .}
\end{figure}

All these constructions are attested in the BNC, the parse trees we
get for all of them seem sensible, and in most cases we can construct
a normalised tree which does the right thing with the specifiers. We
will return to the cases where the normalised tree is wrong later, but
it now seems sensible to consider the interpretation of these trees.

\section{What do specifiers do, anyway?}

Why am I doing any of this? Because I want to do inference. I want to
be able to do things like

\begin{examples}
\item\label{syllogisms}
\begin{examples}
\item
\begin{tabular}{|l|}
\hline
Some accountants are bookkeepers\\
All bookkeepers are crooks\\
\hline
Some accountants are crooks \\
\hline
\end{tabular}

\item
\begin{tabular}{|l|}
\hline
Most swans are white\\
Bruce is a swan\\
\hline
Bruce is probably white \\
\hline
\end{tabular}

\item
\begin{tabular}{|l|}
\hline
All accountants are bookkeepers\\
All bookkeepers are crooks\\
All crooks should go to jail\\
Bruce is an accountant\\
\hline
Bruce should go to jail \\
\hline
\end{tabular}

\item \ldots
\end{examples}
\end{examples}

\subsection{Syllogistic reasoning}

One way to proceed would be to construct a set of syllogistic patterns,
and then try to string these together to make inference
chains. Suppose we decided that \xref{syllogisms:2} was an example of
a sound pattern of inference. We could turn that into a rule by
abstracting away the shared open class words.

\begin{examples}
\item\label{syllogisms:2}
\begin{examples}
\item
\begin{tabular}{|l|}
\hline
All bookkeepers are crooks\\
Simon is a bookeeper\\
\hline
Simon is a crook\\
\hline
\end{tabular}
\end{examples}
\end{examples}

The trees for \xref{syllogisms:2} are as in \fig{syllogisms:3}.
\begin{figure}[ht!]
\hbox{\hspace*{\fill}
\begin{minipage}[t]{0.5\linewidth}

\textcolor{blue}{
 \pstree[levelsep=100pt, nodesep=2pt,treesep=40pt, nrot=:D]{\TR{\begin{tabular}{|c|}\hline$be$\\ \hline\*tense(present,-)\\\hline\end{tabular}}}{
  \TR{\begin{tabular}{|c|}\hline$[bookkeeper>]$\\ \hline\*indefinite\\\hline\end{tabular}}\nbput{predication}
  \textcolor{red}{\TR{\begin{tabular}{|c|}\hline$[accountant>]$\\
      \hline\*universal\\\hline\end{tabular}}\nbput{subject}}}
}

\vspace{0.2in}
\centering{\q{every accountant is a bookkeeper}}
\end{minipage}
\hspace*{\fill}
\begin{minipage}[t]{0.5\linewidth}

\textcolor{red}{
 \pstree[levelsep=100pt, nodesep=2pt,treesep=40pt, nrot=:D]{\TR{\begin{tabular}{|c|}\hline$be$\\ \hline\*tense(present,-)\\\hline\end{tabular}}}{
  \TR{\begin{tabular}{|c|}\hline$[accountant>]$\\ \hline\*indefinite\\\hline\end{tabular}}\nbput{predication}
  \textcolor{blue}{\TR{\begin{tabular}{|c|}\hline$[Simon:NP]$\\ \hline\*name\\\hline\end{tabular}}\nbput{subject} }}
}

\vspace{0.2in}
\centering{\q{Simon is an accountant}}
\end{minipage}
}

\vspace{0.2in}
\hbox{\hspace{1.6in}\begin{minipage}[t]{0.5\linewidth}
\textcolor{blue}{
 \pstree[levelsep=100pt, nodesep=2pt,treesep=70pt, nrot=:D]{\TR{\begin{tabular}{|c|}\hline$be$\\ \hline\*tense(present,-)\\\hline\end{tabular}}}{
  \TR{\begin{tabular}{|c|}\hline$[bookkeeper>]$\\ \hline\*indefinite\\\hline\end{tabular}}\nbput{predication}
  \TR{\begin{tabular}{|c|}\hline$[Simon:NP]$\\
      \hline\*name\\\hline\end{tabular}}\nbput{subject} }
}

\vspace{0.2in}
\centering{\q{Simon is a bookkeeper}}
\end{minipage}}
\caption{Trees for \xref{syllogisms:2}}\label{syllogisms:3}
\end{figure}

\medpara
The \textcolor{red}{red} bits get cut, leaving the
\textcolor{blue}{blue}. Abstracting away the shared elements, this
turns into

\begin{figure}[ht!]
\hspace{0.8in}\begin{minipage}[t]{0.45\linewidth}
\begin{Verbatim}[commandchars=\\\{\}]
([[be, A], \textcolor{red}{arg(subject, *(universal), [B, modifier(every)])}]
  & ([[be, \textcolor{red}{arg(D, E, [B, modifier(a)])}], F]
      => [[be, A], F]))
\end{Verbatim}
\end{minipage}
\caption{Abstraction from \fig{syllogisms:3}: if every B is A and F is
a B then F is A}\label{syllogisms:4}
\end{figure}

\medpara
This would be an interesting way to go. You could do this for all the
syllogisms (including interesting ones like \{\q{Most bookkeepers are
  crooks}, \q{Simon is a bookkeeper} $\vdash$ \q{Simon is probably a crook}\}), and then you could chain through applications of these
patterns to solve a reasonably wide set of problems. Obtaining the
patterns is easy -- just parse the hypothesis and conclusion of a
typical example and then abstract away the shared elements. Writing
the engine is also easy -- if you have sentences whose trees match the
antecedent of some rule then you can infer its conclusion. I would be
tempted to run them forwards whenever I could, because it feels as
though that would be more effective, but I'd have to do some
experimentation to see. 

But I think it would be fairly inefficient, and I think it would feel
rather \textit{ad hoc}. So I'm going to go down a slightly different
route. But I do think it would be an interesting thing to follow
up. Maybe later.

\subsection{Facts and rules}

Most theorem provers for ordinary logic expect you to apply a series
of \defn{normal form} rules which convert from (comparatively)
readable formulae like

\begin{Verbatim}[commandchars=\\\{\}]
exists(A,
       woman(A)
       & forall(B, man(B) 
                   => exists(C, event(C, love)
                                & theta(C, object, A)
                                & theta(C, agent, B) 
                                & aspect(now, simple, C))))
\end{Verbatim}

\noindent
to flatter forms (e.g. quantifier-free form, clausal form, \ldots)
like

\begin{Verbatim}[commandchars=\\\{\}]
woman(#1)
man(B) => event(#2(B), love)
man(B) => theta(#2(B), object, #1)
man(B) => theta(#2(B), agent, B) 
man(B) => aspect(now, simple, #2(B))
\end{Verbatim}

\medpara
The flatter forms make it easier for the inference engine to match
facts and rules: in the original standard form, the fact that this
rule will let you conclude that there was a loving event is buried
deep in the tree, whereas in the flattened form it is immediately
visible and can be used for indexing and matching the rule. Typically
the flatter forms introduce \defn{Skolem functions} (including
functions with no arguments, i.e. constants), i.e. new names for
things that would otherwise be anonymous. 

To do construct normal forms from dependency treees, I will start by splitting
specifiers into \defn{specific} and \defn{general}, and I will further
split specific specifiers into \defn{indefinite} and
\defn{referential}. Things marked by
specific specifiers are fact-like, and will contain terms; things marked by general ones are
rule-like, and will contain variables and some kind of implicational marker.

We will start with a very simple example:

\begin{examples}
\item Some accountants are crooks.
\end{examples}

\begin{figure}[ht]
\hspace*{\fill}
\begin{minipage}[t]{0.45\linewidth}

\pstree[levelsep=100pt, nodesep=2pt,treesep=70pt, nrot=:D]{\TR{\begin{tabular}{|c|}\hline$.$\\\hline\end{tabular}}\nbput{}}{
 \pstree{\TR{\begin{tabular}{|c|}\hline$be$\\\hline\*tense(present,-)\\\hline\end{tabular}}\nbput{claim}}{
  \TR{\begin{tabular}{|c|}\hline$crook>s$\\\hline\*generic\\\hline\end{tabular}}\nbput{predication}
  \TR{\begin{tabular}{|c|}\hline$[accountant>s]$\\\hline\*indefinite\\\hline\end{tabular}}\nbput{subject} }
}
\end{minipage}
\hspace*{\fill}
\caption{\q{some accountants are crooks .}}\label{some accountants are crooks .}
\end{figure}

Even this very simple example raises two problems: what does the
\texttt{*generic} quantifier do, and what does \q{be} mean? We will
postpone consideration of generics for a while and consider an even
simpler case:

\begin{examples}
\item \label{SAIAC}
Some accountant is a crook.
\end{examples}

\begin{figure}[ht!]
\hspace*{\fill}
\begin{minipage}[t]{0.45\linewidth}
\pstree[levelsep=100pt, nodesep=2pt,treesep=70pt, nrot=:D]{\TR{\begin{tabular}{|c|}\hline$.$\\\hline\end{tabular}}\nbput{}}{
 \pstree{\TR{\begin{tabular}{|c|}\hline$be$\\\hline\*tense(present,-)\\\hline\end{tabular}}\nbput{claim}}{
  \TR{\begin{tabular}{|c|}\hline$[crook>]$\\\hline\*indefinite\\\hline\end{tabular}}\nbput{predication}
  \TR{\begin{tabular}{|c|}\hline$[accountant>]$\\\hline\*indefinite\\\hline\end{tabular}}\nbput{subject} }
}
\end{minipage}
\hspace*{\fill}
\caption{\q{some accountant is a crook .}}\label{some accountant is a crook .}
\end{figure}

To get to where we want to be I'm going to do something that looks a
lot like Skolemisation. OK, it \textbf{is} Skolemisation. To get to
there we go through several stages. We start by replacing specifiers
by \defn{in-situ quantifiers} \citep{Milward:94} (see 
\citep{Cooper:83,Keller:87,Vestre:91} for similar
notions), allowing the tense marker to introduce the time at which the
event in question occurred and a reification of the event (that's
quite a lot for it do, but it seems fair enough. What the tense marker
does is to introduce the time at which the event happened,
i.e. introduces and a time and an event and relates them. We will
later on want to add the aspect marker, but this will suffice for now): 
\fig{NF:some accountant is a crook .}. We will refer to trees containing
in-situ quantifiers as \defn{quasi-logical forms} (QLFs) \citep{vanEijck:92}.

\begin{figure}[ht]
\hbox{\hspace*{\fill}
\begin{minipage}[t]{0.45\linewidth}
\begin{Verbatim}[commandchars=\\\{\}]
claim(qq(tense(-)::{present,A},
         at(A,
            qq(indefinite::B,
               {([be,
                   qq(indefinite::{[crook],C},
                      {predication(xbar(v(-),n(+))),C}),
                   qq(indefinite::{[accountant],D},
                      {subject,D})],
                  B)}))))
\end{Verbatim}
\end{minipage}
\hspace*{\fill}}
\caption{QLF for \q{some accountant is a crook .}}\label{NF:some accountant is a crook .}
\end{figure}

\noindent
(note that the in-situ quantifiers in \fig{NF:some accountant is a crook .}
include a mixture of simple quantifiers (\texttt{indefinite::B}) 
and restricted quantifiers
(\texttt{qq(indefinite::\{[crook],C\}}). There's nothing very mysterious
  about these)

We then extract the quantifiers, leaving the quantified variable in
place, and apply them in order of precedence (e.g.\ giving \q{any}
very wide scope): \fig{NF:some accountant is a crook .1}

\begin{figure}[ht]
\hbox{\hspace*{\fill}
\begin{minipage}[t]{0.45\linewidth}
\begin{Verbatim}[commandchars=\\\{\}]
exists(A::{present,A},
       exists(B,
              exists(C::{[accountant],C},
                     exists(D::{[crook],D},
                            claim(at(A,
                                     {([be,
                                         {predication(xbar(v(-),n(+))),D},
                                         {subject,C}],
                                        B)}))))))
\end{Verbatim}
\end{minipage}
\hspace*{\fill}}
\caption{\q{some accountant is a crook .}}\label{NF:some accountant is a crook .1}
\end{figure}

That's not too bad, but scope of the quantifiers seems to be too wide:
\fig{NF:some accountant is a crook .1} says that there is a present
time, and there is an accountant, and there is a crook, and that at
this time the accountant and the crook are in some relationship. It
would seem better if the existence of the time and the crook and the
accountant were all \textbf{inside} the claim, and indeed probably we
would want the existence of the crook and the accountant to be fixed
with respect to the time. We therefore allow \defn{blocking contexts},
which restrict the extraction of quantifiers. Later one we will give
weights to
blocking contexts to specify which quantifiers can escape, but for now
we will just set them to block all quantifiers. That gives us
\fig{NF:some accountant is a crook .2}.

\begin{figure}[ht]
\hbox{\hspace*{\fill}
\begin{minipage}[t]{0.45\linewidth}
\begin{Verbatim}[commandchars=\\\{\}]
claim(exists(A::{present,A},
             at(A,
                exists(B,
                       exists(C::{[accountant],C},
                              exists(D::{[crook],D},
                                     {([be,
                                         {predication(xbar(v(-),n(+))),D},
                                         {subject,C}],
                                        B)}))))))
\end{Verbatim}
\end{minipage}
\hspace*{\fill}}
\caption{\q{some accountant is a crook .}}\label{NF:some accountant is a crook .2}
\end{figure}

We can then convert to quantifier-free form (QFF) (and do a little bit of
tidying-up along the way, including Currying (I think) some
combinations of terms and variables):\fig{NF:some accountant is a
  crook .3}

\begin{figure}[ht]
\hbox{\hspace*{\fill}
\begin{minipage}[t]{0.45\linewidth}
\begin{Verbatim}[commandchars=\\\{\}]
claim((present(#0)
        & at(#0,
             (accountant(#2)
               & (crook(#3)
                   & (be(#1)
                       & (predication(#1, #3, xbar(v(-), n(+)))
                           & subject(#1, #2))))))))
\end{Verbatim}
\end{minipage}
\hspace*{\fill}}
\caption{QFF for \q{some accountant is a crook .}}\label{NF:some accountant is a crook .3}
\end{figure}

\noindent
One last move and then we have this one sorted. \q{Some accountant is a
  crook} will be true if at the present time there is some entity which is an artist and
some (other) entity which is a crook and these two are in fact the
same thing. 

This may feel oddish, but I think there are good reasons for taking it
as the right way to go. Consider, firstly, copula sentences where both
arguments are definite NPs.

\begin{examples}
\item \label{S:COP1}
\begin{examples}
\item
Allan is the one with an earring
\item
The one with an earring is Allan
\end{examples}
\end{examples}

\noindent
Both examples in \xref{S:COP1} pick out two individuals and say they
are the same. The only reasonable way to interpret these two is as
something like \fig{NF:Allan is the one with an earring .}
 -- there is something, \texttt{A}, which is called Allan,
and there is something, \texttt{B}, which has an earring, and these
two things are in fact the same thing. Seems about as reasonable as
anything.

\begin{figure}[ht!]
\begin{Verbatim}[commandchars=\\\{\}]
claim((now=#0
        & at(#0,
             (earring(#2)
               & ref(A,name(A,Allan))
                 = ref(B, (one(B) & modifier(B, ppmod, [with, comp(#2)])))))))
\end{Verbatim}
\caption{\q{Allan is the one with an earring .}}\label{NF:Allan is the one with an earring .}
\end{figure}

\newpage
So for copula sentences where both arguments are definite NPs, we just
want to say that the two things in question are the same. And, as we
have seen in numerous examples, indefinite NPs are usually treated as
involving something like existential quantification. If we maintain
the view that copula sentences with NPs as their objects assert
identity, and that indefinite NPs introduce entities into the
conversation, then for

\begin{examples}
\item
John is a fool.
\end{examples}

\noindent
we get 

\begin{figure}[ht!]
\begin{Verbatim}[commandchars=\\\{\}]
claim((now=#0
        & at(#0, fool(#2)&ref(A,name(A,John),[])=#2)))
\end{Verbatim}
\caption{\q{John is a fool .}}\label{NF:John is a fool .}
\end{figure}

\noindent
What are the consequences of \fig{NF:John is a fool .}? Suppose we had
a rule \texttt{fool(X) => P(X)} for some property \texttt{P}. Then if
the person who satisfies \texttt{ref(A, name(A, John))} is
\texttt{\#36} then \texttt{P(\#36)} will hold. That is exactly what we
want -- to be a fool is to satisfy all the properties that fools
satisfy, and that is what we will get from \fig{NF:John is a fool
.}. In other words, we can get the right consequences of copula
sentences with NP predicates by saying that such sentences say that
their subject and predicate are identical, without forcing a
non-standard interpretation of indefinite NPs.  That means that when I
come to do inference I will need to think about equality, but that's
not going to be the most challenging thing that I face.

Returning to our syllogism, we obtain 
\fig{NF:some accountant is a crook .4} for \xref{SAIAC}.

\begin{figure}[ht]
\hbox{\hspace*{\fill}
\begin{minipage}[t]{0.45\linewidth}
\begin{Verbatim}[commandchars=\\\{\}]
claim((#0=now & at(#0, accountant(#2)&crook(#3)&(#3=#2))))
\end{Verbatim}
\end{minipage}
\hspace*{\fill}}
\caption{\q{some accountant is a crook .}}\label{NF:some accountant is a crook .4}
\end{figure}

\noindent
OK, \q{Some accountant is a crook.} means \q{I am claiming that at a
time which happens to be \texttt{now} there is something which we will
call \texttt{\#2} which is an accountant and something which we will
call \texttt{\#3} which is a crook and that these two things are
actually identical}. That's not an incorrect paraphrase.

Similarly, \q{every accountant is a bookkeeper} comes out as 
\fig{NF:every accountant is a bookkeeper .4}. So if we can deal with
equality we'll be in business.

\begin{figure}[ht!]
\hbox{\hspace*{\fill}
\begin{minipage}[t]{0.45\linewidth}
\begin{Verbatim}[commandchars=\\\{\}]
claim((#0=now & at(#0, accountant(A)=>(bookkeeper(#2(A))&(#2(A)=A)))))
\end{Verbatim}
\end{minipage}
\hspace*{\fill}}
\caption{\q{every accountant is a bookkeeper .}}\label{NF:every accountant is a bookkeeper .4}
\end{figure}

\noindent
\q{I am claiming that at the moment if you show me an accountant I
  will find you a bookeeper who is in fact the same person}

What shall we do about equality? In the antecedent of a rule, it means
that we have to be able to prove that two possibly distinct terms
denote the same thing, which could be tricky if they are complex terms
-- if you know that \texttt{\#1} and \texttt{\#2} are the same entity
and that \texttt{\#3(X)} and \texttt{\#4(X)} are the same functions,
you're still going to have to do some work to determine that \texttt{\#3(\#1)} and \texttt{\#4(\#2)}
 are different names for the same thing. But we can deal with equality
 in the consequent of a rule, or in a fact, simply by merging the two
 names. \fig{NF:some accountant is a crook .4} and 
\fig{NF:EQUELIM}(a) will be true under
exactly the same circumstances, and likewise 
\fig{NF:every accountant is a bookkeeper .4} and 
\fig{NF:EQUELIM}(b). So we can eliminate equality in
positive contexts just by merging the two terms on either side of the equality.

\begin{figure}[ht!]
\hbox{\small \hspace*{\fill}
\begin{minipage}[t]{0.45\linewidth}
\begin{Verbatim}[commandchars=\\\{\}]
claim(at(now, accountant(#2)&crook(#2)))
\end{Verbatim}
\centering{(a) some accountant is a crook}
\end{minipage}
\hspace*{\fill}
\begin{minipage}[t]{0.45\linewidth}
\begin{Verbatim}[commandchars=\\\{\}]
claim(at(now, accountant(A)=>bookkeeper(A)))
\end{Verbatim}
\centering{(b) every accountant is a bookkeeper }
\end{minipage}
\hspace*{\fill}}
\caption{Eliminating equality in positive contexts}\label{NF:EQUELIM}
\end{figure}

\subsection{Queries and Proofs}

So far we have looked at simple declarative sentences. We also need to
handle queries. This is fairly straightforward: where a declarative
sentence has \texttt{claim(\ldots)} wrapped round it by the full stop,
a question as \texttt{query(\ldots)} wrapped round it by the question
mark. The QLF for a query is thus very similar to that for the
corresponding statement:

\begin{examples}
\item\label{Is some accountant a crook?}
Is some accountant a crook?
\end{examples}

\begin{figure}[ht]
\hbox{\hspace*{\fill}
\begin{minipage}[t]{0.45\linewidth}
\begin{Verbatim}[commandchars=\\\{\}]
query(opaque(qq(tense(-)::{present,A},
                at(A,
                   opaque(qq(indefinite::B,
                             {([be,
                                 qq(indefinite::{[crook],C},
                                    {predication(xbar(v(-),n(+))),C}),
                                 qq(indefinite::{[accountant],D},
                                    {subject,D})],
                                B)}))))))
\end{Verbatim}
\end{minipage}
\hspace*{\fill}}
\caption{QLF for \q{is some accountant a crook ?}}\label{NF:is some accountant a crook ?}
\end{figure}

When we convert this to quantifier-free form, we have to note that
questions place their propositional contents inside negative contexts,
so we have to be carefull when removing the quantifiers. What we want
as the QFF for \xref{Is some accountant a crook?} has the indefinite
specifiers replaced by place-holder variables:

\begin{figure}[ht]
\hbox{\hspace*{0.25\linewidth}
\begin{minipage}[t]{0.45\linewidth}
\begin{Verbatim}[commandchars=\\\{\}]
query(at(now, accountant(A)&crook(A)))
\end{Verbatim}
\end{minipage}
\hspace*{\fill}}
\caption{QFF for \q{is some accountant a crook ?}}\label{NF:is some accountant a crook ?:QFF}
\end{figure}

\noindent
\fig{NF:is some accountant a crook ?:QFF} says that I would like 
to know whether there is something which is currently both an
accountant and a crook. Seems like a fair enough interpretation of
\q{Is some accountant a crook?}.

\section{Inference}

We now have representations of sentences expressing facts, rules and
questions. We want to be able to answer the questions by using the
facts and rules, in other words we want an inference engine that will
operate over these representations. There is a wide variety of
mechanisms for carrying out inference -- resolution, tableaux, \ldots
Because we are going to want to do intensional reasoning (reasoning in
a framework that allows quantification over properties and
propositions) we want something which we are confident can be adapted
for such tasks: we will therefore use \citet{Ramsay:01b}'s version of the SATCHMO
inference engine \citep{Manthey:88,Loveland:95} which we have already 
  used for reasoning with a fully intensional logic, namely property
  theory \citep{Ramsay:95a,Ramsay:97e,Turner:87}.

We start with a simple constructive version of SATCHMO (\fig{SATCHMO})
which embodies
the standard natural deduction operators for constructive logic. We
assume that \texttt{not(P)} is a shorthand for \texttt{P => absurd}:
there are no rules explicitly about negation below because anything
that we want to say about negation will be captured in the rules
relating to \texttt{=>}. There are a large number of optimisations \citep{Loveland:91,Loveland:95,Ramsay:91a}
that can be added to this skeletal version of the algorithm, but we
omit these here for clarity.

\newpage
\begin{figure}[ht!]
\hbox{
\hspace{0.3\linewidth}\begin{minipage}[t]{0.45\linewidth}
\begin{Verbatim}[commandchars=\\\{\}]
prove(P, LABEL) :-
    fact(P).
\end{Verbatim}
\end{minipage}
}

\vspace{0.2in}
\hbox{
\begin{minipage}[t]{0.45\linewidth}
\begin{Verbatim}[commandchars=\\\{\}]
%% &-elimination
prove(P, LABEL) :-
    prove(P & Q, LABEL)

%% or-elimination
prove(P, LABEL) :-
    prove(P or Q, label),
    prove(Q => absurd).

%% =>-elimination, modus ponens
prove(P, LABEL) :-
    A => P,
    prove(A, LABEL).
\end{Verbatim}
\end{minipage}
\begin{minipage}[t]{0.45\linewidth}
\begin{Verbatim}[commandchars=\\\{\}]
%% &-introduction
prove(P & Q, LABEL) :-
    prove(P, LABEL), prove(Q, LABEL).

%% or-introduction
prove(P or Q, LABEL) :-
    prove(P, LABEL); prove(Q, LABEL).


%% =>-introduction (conditional proof)
prove(P => Q, LABEL) :-
    assert(P),
    (prove(Q, LABEL) -> 
        retract(P); 
        (retract(P), fail)).
\end{Verbatim}
\end{minipage}
}
\caption{Basic implementation of constructive SATCHMO}\label{SATCHMO}
\end{figure}

Why use SATCHMO? We are going to want do a variety of rather
complicated things -- reasoning with defaults, with epistemic \&
temporal contexts, with fine-grained intensionality, with asymmetric
approximate unification, \ldots So we will accrete extra complexity
and complication to whatever engine we start with. Starting with
something very simple will make it easier to add all this extra
material without becoming swamped by it. The basic implementation of
SATCHMO is startlingly simple -- \fig{SATCHMO} shows the whole
thing. This makes it an attractive starting point for adding
non-standard facilities to.

\subsection{Unification or subsumption?}

The version of SATCHMO presented in \fig{SATCHMO} used unification, as
most theorem provers do, for
matching goals to the consequents of rules. There are, however,
circumstances under which other matching algorithms may be useful.

\begin{itemize}
\item
It may be useful to exploit subsumption relations between terms:
if we have a hierarchy of terms that includes, for instance, that
\texttt{man \sub  human} (e.g. via WordNet) then it may be better to note that \texttt{man(X)
  \sub human(X)} when matching rules than to have a rule that says \texttt{forall(X, man(X) =>
  human(X))} 
\item
Given that the terms in our representations are dependency trees, it
may be appropriate to use a partial matching algorithm which allows,
for instance, modifiers to be ignored so that the term corresponding
to \q{a man with a telescope} in \q{A man with a telescope saw me} can
be matched with the term for \q{a man} in \q{A man saw me}.
\end{itemize}

We therefore replace the original presentation of SATCHMO with
\fig{SATCHMO+subsumption}. Using subsumption instead of unification
does make it more difficult to index rules, since it is no longer
clear when a fact or rule will be useful in a proof -- given the fact
\texttt{man(John)} and the rule \texttt{human(X) => mother(\#1(X), X)},
it is not immediately evident that this rule will let you infer that
John has a mother. We therefore have to be careful about how we store
rules. Nonetheless, using subsumption rather than unification has the
potential to unlock resources such as WordNet, and also to add extra
facilities such as partial matching of subtrees.

\newpage
\begin{figure}[ht!]
\hbox{
\hspace{0.3\linewidth}\begin{minipage}[t]{0.45\linewidth}
\begin{Verbatim}[commandchars=\\\{\}]
prove(P, LABEL) :-
    fact(P'), P' \sub P.
\end{Verbatim}
\end{minipage}
}

\vspace{0.2in}
\hbox{
\begin{minipage}[t]{0.45\linewidth}
\begin{Verbatim}[commandchars=\\\{\}]
%% &-elimination
prove(P, LABEL) :-
    prove(P' & Q, LABEL), P' \sub P.


%% or-elimination
prove(P, LABEL) :-
    prove(P' or Q, label), P' \sub P.
    prove(Q => absurd).


%% =>-elimination, modus ponens
prove(P, LABEL) :-
    A => P', P' \sub P.
    prove(A, LABEL).
\end{Verbatim}
\end{minipage}
\begin{minipage}[t]{0.45\linewidth}
\begin{Verbatim}[commandchars=\\\{\}]
%% &-introduction
prove(P & Q, LABEL) :-
    prove(P', LABEL), P' \sub P,
    prove(Q', LABEL), Q' \sub Q.

%% or-introduction
prove(P or Q, LABEL) :-
    (prove(P', LABEL), P' \sub P);
    (prove(Q', LABEL), Q' \sub Q).


%% =>-introduction (conditional proof)
prove(P => Q, LABEL) :-
    assert(P),
    (prove(Q, LABEL) -> 
        retract(P); 
        (retract(P), fail)).
\end{Verbatim}
\end{minipage}
}
\caption{Constructive SATCHMO with subsumption}\label{SATCHMO+subsumption}
\end{figure}

\subsection{Backwards or forwards?}

You can run an inference engine backwards (to prove \texttt{P}, find a rule
\texttt{P1 \& P2 \& \ldots \& Pn => P} and try to prove each
\texttt{Pi}) or forwards (if you have a rule \texttt{P1 \& P2 \& \ldots
  \& Pn => P} and you know that each of the \texttt{Pi} holds then you
can add \texttt{P} to what you know). Theorem provers are run largely
backwards, but it is reasonable to suggest that people actually do a
degree of forward inference. Consider the examples in \xref{ABSURDS}:

\begin{examples}
\item \label{ABSURDS}
\begin{examples}
\item
My father was born on the moon.
\item
All cats have wings.
\item
I have an infinite number of friends.
\end{examples}
\end{examples}

\noindent
All these sentences are false. What is more, they are patently,
obviously, blatantly false, and anyone reading them will immediately
realise this.

This suggests that you do not just passively listen to what you are
told and store it away for future reference, to be used if it would
help you solve some problem (e.g. answer a question). You actively
assimilate it. There are good reasons for this. When people tell you
things, it is usually because they think that you will do something
with what they have told you. If they shout \q{Fire!}, they expect you
to leave your current location and go somewhere safer. If they say
\q{I could murder a cup of tea}, they expect you to go and make them a
cup of tea. Whatever it is they expect you to do, they generally
expect you to do it now. So you should think about what you have just
been told, and about how it fits into the current situation, because
the odds are that you should do something about it.

The activity of assimilating what you have just been told is often
referred to as \defn{model building} \citep{Bos:01}. I am going to
talk about it in terms of forward inference -- technically there is
little difference between the two (indeed, the standard presentation
of SATCHMO, like many other
theorem provers, proceeds by attempting to build a model of the
premises and the negation of the goal and hoping that this will turn
out to be impossible).

Forward inference is, in general, a very open-ended task. Given the
premises \texttt{\{human(J), forall(X :: \{human(X)\}, exists(Y,
  human(Y) \& mother(Y, X)))\}} there are an infinite number of
  (non-tautologous) conclusions that can be drawn, the effect that J's
  mother is human, and J's mother's mother is human, and J's mother's
  mother's mother is human, and \ldots Assimilating what you have just
  been told involves a bit of forward inference, but not an unbounded
  amount. You could, perhaps, do some kind of unconvincing
  Johnson-Laird-ish kinds of pyschological experiments to try to
  determine just how much; or you could inpose some aribtrary
  restrictions, safe in the knowledge that you can always fall back on
  doing backward inference to solve complex problems as and when you
  need to. For simplicity, I will apply the following strategy:

\begin{itemize}
\item
Add any ground facts that can be added using Horn clauses without
introducing Skolem functions (the only functions we have are Skolem
functions, so this blocks cases like the one above about J's mother
and J's mother's mother). It may be that in certain situations we
should block Skolem functions that are nested to a specified depth,
rather than blocking all such functions.
\item
Add anything that can be added by application of a rule with an
intensional consequent. The key here is that it is almost impossible
to use rules with intensional consequents backwards, because the
consequent will be at best vaguely described and hence it will be
difficult to work out whether it is relevant to a specific goal. To
take a simple case, consider \xref{S:MANAGE}:

\begin{examples}
\item\label{S:MANAGE}
She managed to finish the book she was writing.
\end{examples}

One simple consequence of \xref{S:MANAGE} is that she did indeed
finish writing it -- \q{manage} is a factive verb which asserts, among
other things, the truth of the embedded clause. The obvious way to
capture this is with a rule like \fig{R:MANAGE}.

\begin{figure}[ht!]
\hspace{1.0in}
\begin{minipage}[t]{\linewidth}
\begin{Verbatim}[commandchars=\\\{\}]
manage(X) & xcomp(X, Y) & subject(X, Z) => (Y:Z)
\end{Verbatim}
\end{minipage}
\caption{If Z managed to Y then Z did Y}\label{R:MANAGE}
\end{figure}

\noindent
Allowing this rule to be used backwards would mean that you would try
to use it every time you wanted to prove anything. Using it forwards
means that it will be triggered only when you are trying to assimilate
a sentence about someone managing to do something \citep{Ramsay:01b}.

\item
Add anything that can be added by application of a default rule, so
long as there is no \textbf{current} reason not to. The evidence for
this lies in the double-take that is induced by sentences like
\xref{S:FARMER}(b) 

\begin{examples}
\item\label{S:FARMER}
\begin{examples}
\item
If a farmer owns a donkey he beats it.
\item
If a farmer owns a donkey she looks after it very nicely.
\end{examples}
\end{examples}

\noindent
It is perfectly possible for farmers to be women. Indeed, in 2012 45\%
of farmers in Arizona were
women\footnote{\texttt{https://www.agcensus.usda.gov/Publications/2012/Online\_Resources/Highlights/Women\_Farmers/Highlights\_Women\_Farmers.pdf}}. Nonetheless,\xref{S:FARMER}(b)
seems odd because most people's preconceptions (=
default rules) suggest that farmers are men, so that by the time you
have assimilated \q{a farmer owns a donkey} you will already have
visualised a male farmer, and hence have difficulty dereferencing
\q{she}.

\item
If you can prove the antecedent of a rule with a disjunctive head and
all but one of the disjuncts is provably false then add the remaining
one (subject to the same restriction on Skolem functions as above).
\end{itemize}

These rules will suffice to produce a set of immediate ground
consequences/partial model of an utterance given a context and a set
of background rules; and, given suitable background knowledge, to make
it possible to detect the impossibility of examples like
\xref{ABSURDS} as soon as they are produced.

\subsection{Contexts}

We now extend our representations to enable us to talk about \defn{contexts} -- about, for
instance, temporal contexts such as
the fact that our interpretation \q{Some accountant is a bookkeeper}
says that \texttt{accountant(\#2) \& crook(\#2)} holds at
\scare{now}. We will deal with this and other similar
relations by adding to every formula a specification of the (possibly
nested) context(s) in which it is available. We will do this by adding to
every formula a \defn{label} \citep{Gabbay:89,Gabbay:96}, i.e. a
bundle of information about that formula\footnote{carried in the second
argument of \texttt{prove} in \fig{SATCHMO}: this will eventually turn
out to include all sorts of useful things.}. We will return to this in
Section \ref{SEC:CONTEXTS}. The key notion for now is that we allow a
formula to be annotated with a specification of the context(s) in which
it is available, writing \texttt{P{\IN}C} to mean that \texttt{P} is
available in the context \texttt{C}. Contexts can be temporal (the instant at which something
is true, the interval during which it is true) or epistemic (a
proposition may be available in someone's belief set or in the set of
things they know) or \ldots They do \textbf{not} in general distribute
over logical connectives -- you cannot assume that \texttt{(P \& Q)\IN
  C} = \texttt{((P{\IN}C) \& (Q{\IN}C))} or \texttt{(P or Q){\IN}C} =
\texttt{((P{\IN}C) or Q{\IN}C)}. We will however assume that they nest
-- that \texttt{(P{\IN}C1){\IN}C2} = \texttt{P\IN(C1+C2)}, i.e. if the fact
that \texttt{P} is available in \texttt{C1} is available in
\texttt{C2} then \texttt{P} is available in \texttt{C1+C2}, and we
will always use this rule to flatten nested contexts.

We use this notion to turn \fig{NF:is some accountant a crook ?:QFF}
into \fig{LABELS}(a) and likewise \fig{NF:EQUELIM}(b) becomes  \fig{LABELS}(b).

\begin{figure}[ht]
\begin{center}
\begin{minipage}[t]{0.9\linewidth}
\begin{Verbatim}[commandchars=\\\{\}]
(accountant(A){\IN}[now,believes(hearer)] & crook(A){\IN}[now,believes(hearer)])
\end{Verbatim}

\centering{(a) is some accountant a crook?}
\end{minipage}

\vspace{0.2in}
\begin{minipage}[t]{0.45\linewidth}
\begin{Verbatim}[commandchars=\\\{\}]
(accountant(A)=>bookkeeper(A)){\IN}[now,believes(speaker)]
\end{Verbatim}

\centering{(b) every accountant is a bookkeeper.}
\end{minipage}
\end{center}
\caption{Contexts added as labels: \q{is some accountant a crook?}/\q{every accountant is a bookkeeper .}}\label{LABELS}
\end{figure}

\noindent
\fig{LABELS}(a) asks whether the information available to the hearer
(who is the person who is being asked the question, and hence is the
person who should attempt to use their beliefs to find an answer) supports a
proof that there is someone who is currently an accountant and is also
currently a crook; \fig{LABELS}(b) says that the speaker, who is the
person making the claim, believes that they have access to a rule that
says that is currently the case that if someone is an accountant then
they are also a bookkeeper (note the label applies to the entire
rule).

We now consider what happens in a dialogue with two participants, S and H:

\begin{examples}
\item\label{DIALOGUE:1}
S: every accountant is a bookkeeper.

H: OK

S: every bookkeeper is a crook.

H: OK

S: some man is an accountant.

H: OK

S: is some man a crook?

H: Hmmm, don't ask me, haven't a clue, how on earth would I know?
\end{examples}

After each of S's statements, H says \q{OK}. Why does she do
this? Because all that the statements say is that S believes
something, and there is no necessary requirement for H to believe
something just because S does. There are Grice-ish reasons to expect
that H will believe what S says, but unless she says that she has
accepted S's statements then S cannot be sure that she has. So she
says \q{OK} to indicate that she has, at least tentatively, accepted
what she has been told. We therefore construct a model of H's beliefs
at this point that looks like \fig{FIG:DIALOGUE:1:a}


\begin{figure}[ht]
\begin{center}
\begin{minipage}[t]{0.45\linewidth}
\begin{Verbatim}[commandchars=\\\{\}]
man(#2){\IN}[now,bel(hearer),now]
accountant(#2){\IN}[now,bel(hearer),now]
accountant(A)=>bookkeeper(A){\IN}[now,bel(hearer),now]
bookkeeper(A)=>crook(A){\IN}[now,bel(hearer),now]
\end{Verbatim}
\end{minipage}
\end{center}
\caption{H's beliefs after the first part of \xref{DIALOGUE:1}}\label{FIG:DIALOGUE:1:a}
\end{figure}

\noindent
The context says that is currently true that H believes that it is
currently true that there is someone, who might as well call
\texttt{\#2}, who is a man and is also an accountant, and likewise for
the two rules.

So by the time S asks his final question, H
should have the information she needs to answer it, and (again by some
Grice-ish rules) she should try to do so (well actually by a proper
Grice-ish set of rules she should say \q{Well obviously, why are you
  asking me?}): \fig{FIG:PROOF:1:a}


\begin{figure}[ht]
\begin{center}
\begin{minipage}[t]{\linewidth}
\begin{Verbatim}[commandchars=\\\{\}]
Trying to prove man(_158190)&crook(_158190) in [now,bel(hearer),now]
  Trying to prove man(_158190) in [now,bel(hearer),now]
  Found man(#2) in [now,bel(hearer),now]
  Trying to prove crook(#2) in [now,bel(hearer),now]
  Using bookkeeper(#2)=>crook(#2) to prove crook(#2) in [now,bel(hearer),now]
    Trying to prove bookkeeper(#2) in [now,bel(hearer),now]
    Using accountant(#2)=>bookkeeper(#2) to prove bookkeeper(#2) in [now,bel(hearer),now]
      Trying to prove accountant(#2) in [now,bel(hearer),now]
      Found accountant(#2) in [now,bel(hearer),now]
\end{Verbatim}
\end{minipage}
\end{center}
\caption{H can prove there is a crook}\label{FIG:PROOF:1:a}
\end{figure}

\noindent
Straightforward enough: just a backward chaining proof with a label
that includes the context where things can be proved, with the same
context for the facts that H can use and the query she is trying to
answer. Various things will get more complicated from here on, but
that's our basic engine. If someone tells you something, then if
you're prepared to accept it you should say \q{OK} and add it to your
own belief set; if they ask you a question you should try to answer
it. There are much more complicated things that speakers can do (tell
jokes, be sarcastic, lie, \ldots) and much more complicated things
that hearers can do (believe what they are told, say they don't
believe it, pretend to believe it, \ldots) which we have discussed
elsewhere \citep{Ramsay:06a,Ramsay:06b,Ramsay:08a,Ramsay:09b}; but
this simple model will suffice for now.

Given that our representations now include
contexts, we now need to
modify the specification of SATCHMO to allow for this. We will write
\texttt{C1 << C2} to mean that anything that is available in C2 is
available in C1. Thus, since anything which anyone knows in a context
must be true in that context we can say \texttt{C << [know(\#1) | C]}
for an arbitrary context \texttt{C}.  This version of SATCHMO
is given in \fig{SATCHMO+}. A number of the rules now depend
on properties of the context, since not all contexts support all the
standard introduction and elimination rules (e.g. consider \texttt{A}
and \texttt{B} where \texttt{B} = \texttt{not(B)}: then \texttt{poss(A) \&
  poss(B) $\not\vdash$ poss(A \& B)}, \texttt{nec(A or B) $\not\vdash$
  nec(A) or nec(B)}). We therefore have to attach side-conditions to
the various rules which indicate the kinds of contexts where they apply.

\begin{figure}[ht!]
\hbox{
\hspace{0.3\linewidth}\begin{minipage}[t]{0.45\linewidth}
\begin{Verbatim}[commandchars=\\\{\}]
prove(P \IN C, LABEL) :-
    fact(P' \IN C'), P' \sub P, C << C'.
\end{Verbatim}
\end{minipage}
}

\vspace{0.2in}
\hbox{
\begin{minipage}[t]{0.45\linewidth}
\begin{Verbatim}[commandchars=\\\{\}]
%% &-elimination
prove(P \IN C, LABEL) :-
    prove((P' & Q) \IN C, LABEL), 
    P' \sub P, C << C'.

%% or-elimination
prove(P \IN C, LABEL) :-
    prove((P' or Q) \IN C', label), 
    C << C', P' \sub P,
    prove((Q => absurd) \IN C''),
    C << C''.

%% =>-elimination, modus ponens
prove(P \IN C, LABEL) :-
    (A => P') \IN C', 
    P' \sub P, C << C',
    prove(A \IN C'', LABEL),
    C << C''.
\end{Verbatim}
\end{minipage}
\begin{minipage}[t]{0.45\linewidth}
\begin{Verbatim}[commandchars=\\\{\}]
%% &-introduction
prove((P & Q) \IN C, LABEL) :-
    prove(P' \IN C', LABEL), P' \sub P, C << C',
    prove(Q' \IN C'', LABEL), Q' \sub Q, C << C''.

%% or-introduction
prove((P or Q) \IN P, LABEL) :-
    (prove(P' \IN C', LABEL), P' \sub P, C << C');
    (prove(Q', LABEL \IN C''), Q' \sub Q, C << C'').



%% =>-introduction (conditional proof)
prove((P => Q) \IN C, LABEL) :-
    assert(P \IN C),
    (prove(Q \IN C, LABEL) -> 
        retract(P \IN C); 
        (retract(P \IN C), fail)).
\end{Verbatim}
\end{minipage}
}
\caption{SATCHMO with contexts}\label{SATCHMO+}
\end{figure}

The first three rules of \fig{SATCHMO+} just say that you can
prove something in a context \texttt{C} if you can prove it in
\texttt{C'} where anything which is available in \texttt{C'} is
available in \texttt{C}. The fourth just unwraps \texttt{not(P)} in
the standard manner of constructive logic. The last rule says that you
can prove \texttt{P => Q} in \texttt{C} by assuming \texttt{P} is
available in \texttt{C} and trying to prove \texttt{Q} in the same
context. This is not, however, always legitimate -- the argument above
showed that in general \texttt{(P => Q){\IN}C} entails \texttt{P{\IN}Q =>
  Q{\IN}C}. We do not, however, in general have the converse. We will
refer to contexts in which \texttt{P{\IN}Q => Q{\IN}C} entails \texttt{(P =>
  Q){\IN}C} as \defn{closed} contexts. Conditional proofs of the kind
embodied by the final rule in \fig{SATCHMO+} can only be
carried out in contexts which are known to be closed.


\newpage
\bibliography{refs}

\appendix
\newpage
\section{\q{all of the \ldots}, \q{all the \ldots}, \q{the \ldots
    were all \ldots}}\label{APP:ALL}

Examples of constructions involving \q{all} from the BNC.

\begin{examples}
\item
\begin{examples}
\item who had been hostile , in print , to all of the participants in the historical events which supplied part of his 
\item anything about , but would be willing to dismiss along all of the social sciences , I was in danger of being irrevocably 
\item This mixed condition he shares with many others , not all of them writers ; it is a condition we are entitled to 
\item organs of the body may be affected , and when all of these factors are combined the results can be severe , with 
\item , ' said Thomas , ` which master announced to all of them was faulty , and that was why Rover slept there 
\item description ' which most alarmed them ; for as almost all of them wistfully pointed out : ` it shows exactly how we 
\item with the sights , sounds , smells and tastes --all of them clamorous and variegated and , not least , the girls 
\end{examples}

\item
\begin{examples}
\item coalesced , inevitably , in the act of painting when all the discrete , scattered moments , followed up , caught on 
\item compiler of a catalogue raisonn\'{e} will have seen and compared all the works listed , or will scruple to state if some 
\item ` tenderised ' -- a word Roth likes , for all the awkwardness it imparts to the operations to which it refers 
\item I would like to thank you and your Team for all the effort and resources you have put into providing a home 
\item the other people who figure in his books , wrote all the time about himself , both in autobiographical and in fictional 
\item a taxable income at equal to the gross amount of all these payments, as the Gift Aid payment being contemplated , 
\item the form requires the donor to state that he satisfies all the conditions relating to Gift Aid ( as which , see 
\end{examples}

\item
\begin{examples}
\item a hillside over Lake Garda , the lakeside promenade and the beach are all within easy walking distance , even it 's an uphill 
\item , St John Ambulance men , special constables , and the like were all used to define them as being somewhat ` unreal ' 
\item a beard , the diadem , the upturned eye and the hairstyle are all derived from that of the great conqueror Alexander the Great 
\item 's announcement that its development-area status is being removed , the staff are all cowering under their desks while the population throw bricks through 
\end{examples}
\end{examples}

\end{document}