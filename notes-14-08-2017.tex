
\documentclass[11pt]{article}
\evensidemargin=0in 
\oddsidemargin=0in
\textwidth=6.5in
\topmargin=-1.0in
\textheight=10.5in
\usepackage{defns}
\usepackage{natbib}
\bibliographystyle{elsart-harv}

\begin{document}
\title{More thoughts on determiners}\date{}
\maketitle

This is an extended rumination on the syntactic properties of
determiners. Once I'm comfortable with the structure I will return to
the semantics, and to writing a program that embodies all of this. But
for now I'm just thinking about syntax. 

This rumination is pretty informal, and leads through a series of
suggestions, observations and provocations. What I need to know right
now is

\begin{itemize}
\item
where do I say something that is transparently stupid?
\item
what should I read to see other people who have said the same thing?
\item
what examples have I missed? I'm extracting examples by running
regexes over the BNC. This will tell you about whether the examples
you make up in your head occur in freely occurring text, but it won't
prompt you to think of things you haven't thought of. So any ideas on
what I \textbf{haven't} looked at would be very welcome.
\end{itemize}

As just noted, the examples are obtained by running regexes on the
BNC, which is 100 million words and is the largest corpus I have
access to (I am aware of one larger one, namely the TenTen corpora --
$10^{10}$ words per language, but I haven't got and can't get a
copy). But 100 million is enough to provide some useful
information. Regexes are a fairly blunt instrument, and they produce
quite a lot of false positives. False negatives don't worry me -- if
something doesn't occur in the BNC then it is vanishingly rare (less
than 1 in $10^8$). False positives aren't a big theoretical issue,
because so long as there are some real positives then the phenomenon I
am interested in does occur, but I do remove them manually because
they clutter up the argument.

\subsection*{DET + of + NP(+def)}

We start with the observation that a wide range of quantifiers
(e.g. \q{many}, \q{some}, numbers) can combine
with \q{of NP(+def)}, and that this is fairly common (17\% of occurrences
of these quantifiers).

{\small\begin{verbatim}
                we just have to ask how *** many of *** our congregation have been added during
           provide home care for one in *** four of *** all those dying with AIDS in
                  at home for up one in *** four of *** all those who died of AIDS
                              Brian was *** one of *** the very first members of the
             These factors help explain *** some of *** the reasons why the total number
n necessary office work and telephoning *** some of *** our low-dependency clients to see how
                            We make the *** most of *** this and scoot off to the
    Cedria is a full-time volunteer and *** one of *** national network of volunteers .
sexual intercourse between two people , *** one of *** whom is infected .
                                    For *** most of *** them there is a cure
\end{verbatim}}

There's nothing monstrously weird about this. It's how you do
quantification over known sets, and while it's not obvious (to me) why
you need to include \q{of} as a case-marker, it's no more unobvious
than why you have to use it in things like \q{a piece of cake}. At any
rate, I'm not going to worry about it.

When you come to look at \q{all}, the case-marker becomes optional: 

{\small\begin{verbatim}
              Interest was expressed in *** all of the *** organisational aspects of home care including
                          This can make *** all the *** difference to someone who feels unsafe
           a single agency can not give *** all of the *** support required .
       donor to state that he satisfies *** all the *** conditions relating to Gift Aid (
                          &bquo &hellip *** all the *** residue of my estate &equo .
organisations to ensure that people get *** all the *** care they need in the way
                    &bquo Thank you for *** all the *** help you gave to us and
\end{verbatim}}

Cases without \q{of} are quite a lot more frequent (this would have
been evident
if I had included more examples, but that would have been too many
examples to look at), but I can't see any
difference in meaning, and I could include \q{of} in nearly all the
cases above that don't have it and omit it from the ones that do,
without changing the grammaticality or the meaning.

This doesn't happen with any of the other determiners that take \q{of}:
here are \textbf{all} the occurrences of these determiners with an NP beginning
with \q{the} in the A section of the BNC (16 million words). In almost
all of them, the determiner has an elliptical N (see below), and the \q{the} is the
start of the following NP; and the others feel strongly ungrammatical
\q{some the best examples of the Gibbons technique}, \q{the scene of one
the gravest blows to democracy carried out}. None of them seem to me to
involve complex determiners of the kind we had with \q{all the
  things}, \q{all the works}, \ldots above.

{\small\begin{verbatim}
              problem , however , it is *** one the *** novel shares with many of the
           the travelling to get to the *** one the *** night after and seeing a different
            work you have done made you *** any the *** better ?
                        One the glass , *** two the *** show , three the end .
                                 One by *** one the *** members of the crowd trickled out
jective interpretation of events is not *** one the *** institution of policing is geared up
          Now , he concluded gloomily , *** some the *** best examples of the Gibbons technique
tutionally established in America , and *** one the *** tendency towards which in this country
                                 One by *** one the *** gerbils were scooped up , turned
        real winner , though &mdash the *** one the *** highlighted products of London 's solariums
               to gloat at the scene of *** one the *** gravest blows to democracy carried out
       difficult for any company , even *** one the *** size of BAe , to go
                               Not even *** one the *** size of GM , which has
                   by the KGB , and for *** some the *** advertisements confirmed their darkest fears .
otally different situation now from the *** one the *** Government inherited . &equo
             number of votes , while in *** two the *** Greens did better .
              they are of Welsh slate , *** many the *** work of Madge Whiteman .
          deserve to be better known as *** many the *** months that elapsed proved fatal .
                      After a minute or *** two the *** Asian suddenly slid out from the
     the sexual instinct &bquo is never *** any the *** original object but only a surrogate
               in 1989 , to restrict to *** two the *** number of parties allowed to fight
                  for many hours on day *** one the *** river seemed amazingly narrow , one
                                 One by *** one the *** dogs on the left of the
                                 One by *** one the *** Corporals who commanded each rig section
      James Whittaker , that anyone was *** any the *** wiser .
                  &bquo But if there is *** one the *** people of Birkenhead will support him
                                   With *** some the *** feet .
                                   With *** some the *** shoulders .
      , even foreseeable , obscured for *** most the *** suicide of the art .
                   So he turned up with *** one the *** following day .
\end{verbatim}}

So \q{all} does something no other determiner does: it allows you to
drop the case marker when it combines with a NP(+def). Note that not all
NP(+def)s have an explicit determiner. Pronouns are also +def, but in
this case \q{all} follows the pronoun.

{\small\begin{verbatim}
          ( &bquo no the characters say *** they all *** speak in some version of her
              or genuine life , has for *** them all *** the aspects of a hallucination .
                its aims , matter , and *** they all *** feel an excitement about the job
             sooner or later it gets to *** them all *** , even Hilda , love at
               I have been able to give *** them all *** a good education and every chance
            And that is perhaps the way *** they all *** want it .</hi>
               me I should respond , as *** they all *** told me my true nature ,
                                    Are *** they all *** cottars and tinkers ?
\end{verbatim}}

I think that these examples all do universal quantification over the
plural set denoted by \q{they}/\q{them}, e.g. that if we replace
\q{they} by \q{the people} then\q{Are they all cottars
and tinkers ?} is pretty much the same as \q{Are all the people cottars
and tinkers ?}. 

It's not quite as simple as that, because when \q{all} is associated
with \q{they} then it can (must?) follow the adjacent verb if it is an
auxiliary or a copula:

{\small\begin{verbatim}
                None of them is new and *** they are all *** straightforward .
                  Over a period of time *** they will all *** be ill .
                      The point is that *** they are all *** vital young men with love on
                                    And *** they 'd all *** packed them up this morning .</p>
     would look at that noticeboard and *** they would all *** understand the implications behind what he
fered alternative accommodation , which *** they were all *** sensible enough to accept .
                    a risk , and a risk *** they could all *** get by without taking .
               reminded her of me ) and *** they were all *** dumped in my front hall .
       pain arising from injuries , but *** they are all *** banned .
                                    And *** they are all *** widows now .
\end{verbatim}}

Once we recognise that this can happen, we can have a look at other
examples where \q{all} follows a definite NP. My regexes here do produce
quite a lot of false positives, but I am pretty confident that in the
examples below you could replace \q{NP all} by \q{all NP} and retain the
same truth conditions.

{\small\begin{verbatim}
   poverty and the availability of hand *** guns all play *** their part .
         their second message , and the *** fuss all died *** down , but he 's always
           of that term , my final-year *** class all came *** up my office to give me
                                    The *** boys all liked *** him , too &mdash and some
      to give an undertaking that their *** entries all comply *** with age ceilings .
             in which the ladies of the *** chorus all wore *** wellington boots .
\end{verbatim}}

I am \textbf{not} saying that these are the only ways that \q{all} gets used, but
they do look to me like a very closely linked group: \q{all} means every
member of the set/property denoted by the specifiee, if the specifiee
is \texttt{+def} then you may, but don't have to, choose to mark it with \q{of};
and if you don't mark it with \q{of} then you can, and in some
circumstances must, right-shift \q{all}. Note that something similar-ish
happens with \q{each}, where you do mark the target with \q{of} if it's an
accusative pronoun but you just right-shift it if it's a subject-case
one. These are quite a lot rarer than ones with \q{all}, but I think
that that's just because \q{each} is rarer than \q{all}. 

{\small\begin{verbatim}
                 as some would see it ) *** they each *** had for the Tradition &mdash at
                   In some older dances *** they each *** hold a corner of a handkerchief
          the amount of time and effort *** they each *** put into the deal .
            on its books and is sending *** them each *** a booklet when their policies are
                                Clearly *** they each *** believed there was a real issue
\end{verbatim}}

Note that \q{each} and \q{all} both occur with the \q{of NP(+def)} form as
well:

{\small\begin{verbatim}
     January and March this year &mdash *** all of them *** long-term detainees held without charge or
          shares with many others , not *** all of them *** writers ; it is a condition
           areas or &bquo rooms &equo , *** each of them *** having a different function or theme
             one of them senses smoke , *** all of them *** will sound an alarm .
             leaf by leaf , then patted *** each of them *** carefully with a tea towel .
        possible in the seventies , not *** all of them *** successful &hellip
                                   What *** all of them *** contend is that the auditory effect
\end{verbatim}}

Again I can see barely any difference between \q{all of them} and \q{them
all}: \q{She threw all of them away except one} and \q{She threw them all
away except one} are virtually interchangeable. The last one is
interesting: this is how you do an \q{of NP(+def)} version of the
subject -- \q{What all of them contend is \ldots} = \q{What they all contend is
\ldots}. Likewise \q{all of them will sound an alarm}
= \q{they will all sound an alarm}.

I don't think that \q{all} and \q{each} in examples like \q{they are
  all vital young men} and \q{they each believed there was a real
  issue} are adverbials: the parellels with \q{all of them are vital
  young men} and \q{each of them believed there was a real issue} make
me discinclined to believe this, as does the fact that \q{all} in
\q{it gets to them all} has to be linked to \q{them}. \q{all} in these
sentences isn't just some floating adverbial, it's a specifier on the
preceding NP.

\paragraph{Summary to this point:} a lot of determiners can combine with a simple
NN or a phrase of the form \q{of NP(+def)}. In the cases of \q{all} and
\q{each}, you
can drop the \q{of} from this construction, and in that case you may
(and sometimes must) shift the \q{all}/\q{each} to follow its target.

\subsection*{Ellipsis}

Determiners normally determine something. There are, however, fairly frequent
cases of things that you would normally regard as determiners with no
following NN. Again, my regexes for this are a bit flaky, but here's a
pile of examples:

{\small\begin{verbatim}
   in military or police custody &mdash *** some executed *** without trial and many others as
      is internationalism , a feature , *** some might *** say , of twentieth-century art ;
               of art , or aesthetics , *** many do *** not in fact contain any art
          possible , so we apologise if *** any are *** missing , and guarantee , they
   but it is increasingly unlikely that *** any would *** be acceptable to the United board
       , another was better organised , *** two had *** marvellous raw materials , another was
            in Canada from my village , *** four came *** back .
\end{verbatim}}

There are some others which are so common that we've more or less
come to assume that there are two distinct words: we say that \q{this} and
\q{that} can be either determiners or pronouns, and that \q{one} can be a
a determiner or a noun. But once we accept that \q{many}, \q{some},
numbers bigger than one, \ldots can appear without a noun, then it looks
as though maybe \q{this} and \q{that} are also noun-less determiners
rather than pronouns. \q{One} can appear as all sorts of apparently
different things -- I may return to that later. I've only done
examples with \q{this} below, because \q{that}, like \q{one}, is so
multipurpose that we get loads of other readings which will just
confuse things. I quite like \q{For some this makes
hospitalization inevitable}, because it also contains an elliptical
use of \q{some} as well.

{\small\begin{verbatim}
    helpful in identifying the need for *** this service *** .
             , bquo With the opening of *** this office *** in Glasgow , ACET 's volunteer 
not all employers offer their employees *** this facility *** ) .
              or dying with AIDS , even *** this is *** increasingly common . equo
                               For some *** this makes *** hospitalization inevitable .
            specified in the Deed , and *** this is *** the sum that is payable each
\end{verbatim}}

\paragraph{Summary}: at least some things which are normally thought of as
determiners can appear without a following NN. Just to make matters
even more challenging, some of these can also include a phrase that
you would normally think of as an NN post-modifier:
determiner+0+relative clause, determiner+0+PP (always a post-modifier:
if you wanted a premodifier you'd say \q{a long one}, \q{a simple one}, \ldots

{\small\begin{verbatim}
                     Indeed , there are *** some who *** feel that there should be no
               While the party contains *** many who *** actively seek peace and reconciliation ,
              The Conference would wish *** all who *** may take part in the referendum
ghly organized pressure group headed by *** many who *** had previously mounted the Pro-Life Anti-abortion
             His words were taken up by *** many who *** would not have dreamed of opening

  Military Prison outside Kuwait City , *** some for *** over a month , reportedly in
         , lined with Baroque statues ( *** many between *** 1700 and 1720 ) , and
                 in a number of books , *** some by *** investigative outsiders , some in memoirs
                  state , it is seen by *** many in *** the alliance to inhere in the
    dedicated pastors and much liked by *** many in *** the local community , immediately opposed
\end{verbatim}}

\subsection*{Semi-determiners}

It is fairly common practice to distinguish between adjectives and
specifiers/determiners. \q{red} is an adjective, i.e. is a word which adds some
information to the description supplied by the head noun; \q{the} is a
determiner, i.e. it tells you what to do with this description (e.g. find
an entity which can be proved to fit the description using only
information that is in the minutes). We know that for some kinds of
NP, you don't have to have a specifier (\q{I was playing tennis},
\q{She prefers peaches to pears}): there is a lot of discussion of
what you are supposed to do with the descriptor in these situations,
almost all of it wrong (\citet{Ramsay:92b} gets it right, everyone else
gets it wrong), but it is clear that there is some kind of
implicit specifier here. And that sort of already says that you can
add information to an NN/NP which already as a specifier -- \q{I was
playing really rubbish tennis}, \q{She prefers ripe peaches to
underripe pears}.

Well, it sort of already says that, but of course the specifier on a
bare noun like \q{tennis} or \q{peaches} is only implicit, so maybe
that's why you can add an adjective to it. There are, however, some
words that would, again, normally be regarded as determiners but which
can be preceded by explicit determiners. I'm just going to include
some cases with \q{two} and \q{many} here, because other examples tend
to be less clearcut and include more false positives. I will do some
later, but for now I'm just offering these as examples to show that
there are cases where things that are normally regarded as specifiers
can appear in more adjective-like positions: 

{\small\begin{verbatim}
ard planning and good communication are *** the two *** foundation stones that must be in
         help and perhaps I can combine *** the two *** visits .
 first hand information is generated by *** the many *** missions and research trips Amnesty sends
      even the functions and methods of *** the two *** sorts of writer have drawn apart
                         This is one of *** the many *** books which address the snobbery of
\end{verbatim}}

We already know that there is an order in which simple adjectives can
occur -- that you can say \q{a big red bus} but not a \q{a red big
bus}, \q{a fine young lamb} but not so easily \q{a young fine lamb} --
and one simple way to deal with this is by assigning a \scare{strength}
to each adjective, requiring them to be attached in order of strength
(this is a fairly yuk way of doing it, and it should really be something
about either the permanance or the discriminatoriness of the property
described by the adjective, but it's a useful working approximation).

So I am now going to say that there are two things we care about:

\begin{description}
\item\textbf{modified:} an NN can have modifiers attached to it. The
modifier will assign a number to the feature \texttt{modified}, with
the constraint that you can't add a modifier to a phrase which already
has higher value modifier attached to it.

\item\textbf{specifier:} a phrase has a specifier if it consists of a
description and an instruction telling you what to do with that
description (er, this is a very proof-theoretic way of looking at it,
which of course is what I like. It could be rephrased model
theoretically, e.g. by saying that a specifier takes an entity of type
$e \implies t$ and turns it into something of type $(e \implies t)
\implies t$ (e.g. \q{a} = $\lambda P \lambda Q \exists X(P.X \and
Q.X)$, where $P$ and $Q$ are themselves properties/functions from $e
\implies t$)). 
\end{description}

Given that, we can say that a word can modify something headed by an N
so long as it hasn't already been modified by something stronger; and
that certain bare nouns come with a built-in specifier, and certain
modifiers can supply one. \textbf{There is no strong link between
  these}, in particular a modifier can add a specifier to something
that already has one. So in \q{the two foundation stones}, \q{stones}
has a built-in specifier; this is inherited when the adjective
\q{foundation} is added; \q{two} is a stronger modifier than
\q{foundation}, so it can be added, doing two things: it improves the
description by saying how many foundation stones there were, and it
adds a specifier; \q{the} is stronger yet, and its job is to add a
different specifier without enriching the description. All of \q{
  planning and good communication are stones}, \q{ planning and good
  communication are foundation stones}, \q{planning and good
  communication are two foundation stones}, \q{ planning and good
  communication are the two foundation stones} are grammatical and
meaningful (well, the first one's a bit odd): as we add modifiers we
get to know more about these stones, and we get different instructions
about what to do with this description (do what you do with bare
plurals for \q{stones} and \q{foundation stones}, introduce some
stones into the minutes with \q{two foundation stones}, find a pair of
foundation stones in the minutes with \q{the two foundation stones}).

\paragraph{Summary:} some words which modify phrases headed by Ns
include an instruction about what to do with the description encoded
by the target phrase. These instructions can be overridden by
\scare{stronger} modifiers.

\subsection*{Special cases}

A lot of N modifiers display very idiosyncratic behaviours. With any
luck, the patterns described above can be used to explain, though
probably not predict, these behaviours. The current section looks at a
number of such cases.

\subsubsection*{\q{few}}

\q{few} accepts elliptical targets (first four below) and combines
with \q{of} (next four). Nothing
so odd here.

{\small\begin{verbatim}
                   down a word or two , *** few will *** turn as naturally to painting or
           &equo common in such chefs , *** few are *** offered jobs .
           baskets for the summer , but *** few arrange *** for a similarly splendid winter and
                 And of those that do , *** few will *** try them more than once or

                           There can be *** few of *** us who can not make a
se-pipe will disperse larger colonies , *** few of *** the insects surviving to climb back
committed organic gardeners , there are *** few of *** us who never need to resort
      autocracy can produce , but which *** few of *** the chief officers seem willing to
\end{verbatim}}

It can also combine with a preceding \q{the}, just like
\q{many}. Uniquely (I think) it can combine with a preceding \q{a}:

{\small\begin{verbatim}
               for the format is one of *** the few *** in which a generous number of
ropologist Jacques Maquet knew was that *** a few *** weeks after finishing this sombre painting
            who could not wait to climb *** the few *** steps to the communal toilet on
                     Representatives of *** the few *** people they had come to know
        shouts of laughter broke out as *** a few *** of them came out from the
                  They had made sure of *** the few *** proprietors between there and Dunkeld ;
\end{verbatim}}

\q{a few} and \q{the few} do appear with \q{of NP[+def]}, but they are
pretty rare -- 3\% (24/719) of the occurrences of \q{a/the few} in \texttt{BNC/A}.

{\small\begin{verbatim}
        shouts of laughter broke out as *** a few of *** them came out from the gate
                    to the end , one of *** the few of *** her generation who integrated the immense
            crack of the 15 pounders as *** a few of *** them opened fire , the guns
                                     As *** a few of *** us stand and look at the
\end{verbatim}}

\q{few} does also display a couple of other unusual
behaviours. Firstly, it accepts a small range of adverbs as modifiers:

{\small\begin{verbatim}
                                 I have *** far fewer *** friends and I am partially sighted
  the illustrations in any general book *** relatively few *** are in colour ; a careful
        who had any career success were *** extremely few *** in number .
            my own life there have been *** very few *** genuine beginnings , only three or
      listed &mdash and there are still *** dismally few *** &mdash such an attitude inevitably leadsd
                              There are *** surprisingly few *** popular patterns of cutlery &mdash most
\end{verbatim}}

Secondly, \q{a few} co-occurs very frequently with \q{only}:

{\small\begin{verbatim}
gesting visits at breakneck speed where *** only a few *** items or rooms will be seen
rmation about sitters for portraits are *** only a few *** of the varied topics which can
      terms of reference which can give *** only a few *** useful results , for on the
                               It takes *** only a few *** minutes .
   to remember events that had happened *** only a few *** minutes earlier .
          country , may be contrived in *** only a few *** years in a garden by reversing
\end{verbatim}}

\q{the few} does also occur, though very very rarely: seven instances
in the entire BNC. Some are elliptical, or elliptical with a
post-modifier, but I don't think there's a huge amount to be read into that.

{\small\begin{verbatim}
   extended to all , essentially affect *** only the few *** ; and now an attempted prohibition
               to live out my life with *** only the few *** possessions I have managed to buy
                  had no voice at , and *** only the few *** remaining Polish gentry had any representation
             oil and orange juice , but *** only the few *** scheme-housing children had bathrooms .
               , at the highest level , *** only the few *** are competent .
                  in so low a tone that *** only the few *** near them had heard him .
                    Up now I have heard *** only the few *** reluctant words in the lane .
\end{verbatim}}

I'm not at all sure that there is any big connection between \q{only}
and \q{a/the few}. I think that it is probably a sentence modifier,
probably along the lines suggested by \citet{Ramsay:94a}. I will
return to this to see how \q{only} combines with other specifiers.

\subsubsection*{\q{least}, \q{most}}

\q{least} and \q{most} share a number of behaviours. Most of these are
fairly straightforward, and I will go through these first to eliminate
them before we get to the interesting ones.

The first, and most obvious, thing is that \q{most} can behave as a fully-fledged specifier:

{\small\begin{verbatim}
          town as the defendant &equo ( *** most blacks *** live in the same part of
                           In reality , *** most attorneys *** have made almost no preparation for
              interests , as is true of *** most groups *** of artists , but their initial
                    It will be clear to *** most people *** here that the attack is deserved
                  , and are still , for *** most people *** , including themselves , palpably Jewish
\end{verbatim}}

Some people think that \q{most attorneys have made almost no
  preparation} means \q{more than half the set of attorneys have made
  almost no preparation}, some think it's a rule that allows you to
infer \q{X has made almost no preparation} from the assumption that X
is an attorney so long as you have no evidence that he or she has
actually made some preparation. They are both tenable positions, and I
am not concerned (here) with which, if either, is right. I am simply
taking these as examples where \q{most} is a specifier. The regex that
returned me these examples does \textbf{not} return anything
comparable for \q{least}. I do not believe that \q{least} can be used
in this way.

Then they can both function as straightforward adjectives, meaning
something like \q{smallest} and \q{greatest}: you could more-or-less
substitute \q{smallest} and \q{greatest} for \q{least} and \q{most} in
the folloing examples and get something that was almost grammatical
and meant almost the same. These uses just about always occur with
\q{the}: I found no examples of either of these being used as an
adjective with \q{a}. This shouldn't be surprising -- there can only be one thing
that is at the extreme end of a scale, so uniqueness is pretty well guaranteed.

{\small\begin{verbatim}
      wisest of our ancestors never had *** the least conception *** of any of 'em .
bquo something exquisitely fresh , with *** the least amount *** of modification in the process of
at the constructivism that has received *** the most attention *** in psychology and philosophy has been
 their wretched clothes &hellip without *** the least sign *** of sympathy . &equo
                       But if there was *** the least chance *** of getting out before bedtime he
\end{verbatim}}

They can also both be used as adverbs, generally but not always for
modifying adjectives (or adjectival gerunds (gerundives? I can never
get this straight)). There are occasional uses as straight VP
modifiers (see fourth group) but they are much rarer. This time you do
get indefinite NPs (third group), which seems odd at first sight,
because again the adjectival phrase (\q{most boring}, \q{most
  frightful}) puts the entity at the extreme end of some scale, and
hence uniqueness would seem to be guaranteed.

{\small\begin{verbatim}
           with other agencies , is the *** most effective *** way of ensuring the needs of
                                    The *** most common *** way for the virus to spread
                                  It is *** most important *** to appoint at one Executor when
    this global epidemic wherever it is *** most needed *** .
\end{verbatim}}

{\small\begin{verbatim}
           : Making it</hi> must be the *** least pious *** book that has ever been written
                                    The *** least expensive *** method is to root your own
                        But even in the *** least conducive *** conditions , with no wind ,
aining the greatest efficiency with the *** least possible *** expense and labour
\end{verbatim}}

{\small\begin{verbatim}
             as Harsnet had written ) , *** a most boring *** subject .
            in the morning and then had *** a most frightful *** pain in her tummy . &equo</p>
                  , the peace lily , is *** a most elegant *** houseplant , with glossy green lance-shaped
             man of God to officiate at *** a most solemn *** sacrament &equo . &equo
\end{verbatim}}

\noindent(there are no examples of this with \q{least})

{\small\begin{verbatim}
              at the very heart of what *** he most wished *** to believe .
       of little flights of steps where *** he least expected *** them .
 Wimbledon remains favourite as the one *** they most want *** to win ) are due very
      speak &mdash to those steps which *** they most enjoyed *** and thus performed best , they
           who could not say the things *** he most wanted *** to say but who , nevertheless
\end{verbatim}}

I think that the above are all the straighforward uses of \q{most} and
\q{least}. But they also take part in the construction that started
this ramble through the thickets of determiners and specifiers --
\q{at (the (very)?)? least/most N}:

{\small\begin{verbatim}
quo every Parishioner shall communicate *** at the least three *** times in the year , of
                               There 's *** at least one *** track from all eight albums ,
               Hall on perhaps two , or *** at the most three *** , separate occasions simply for the
             the same basic source , or *** at the very least two *** sources .
                       There 's usually *** at least two *** of them .
                               There 's *** at least one *** blown over is n't there ?
\end{verbatim}}

These are fairly rare (these are all the instances of any variation on this pattern
in the BNC). They all seem to have much the same truth conditions:
a Montague-style translation would be $\lambda P \lambda Q (\exists X :: \{|X| \ge N
\& P.X\} Q.X)$ -- I can't see that the presence of \q{the} makes any
difference at all, and \q{very} is commentary rather than
propositional content. The last one is interesting, since it seems to
me to be another example of a quantifier where some element has been
displaced -- I think that \q{Only send in three tracks at the most
 ( naturally your best )} and \q{Only send in at the most three tracks
 ( naturally your best )} say the same thing, with \q{at the most'}
having been displaced for some kind of discourse reason.

If we don't demand that \q{at (the (very)?)? least/most} is followed
by a number, we get the following:

{\small\begin{verbatim}
 disappointing for a reader who expects *** at the least *** some evaluation of the shows .
             are unhappy here , that 's *** at least *** partly my fault .
          This makes the anthropologist *** at the least *** an uncomfortable associate , at the
      evening to realise that there was *** at the very least *** and latest , an unfinished conversation
                               There 's *** at least *** an hour or so of sunshine
         is a cognisant act involving , *** at the very least *** , a conception of a one's
       has a rather privileged status ; *** at the very least *** , that it has a beginning
              have been revealed ; or , *** at the very least *** , and least interestingly , that
            meal in a restaurant , then *** at the very least *** we should all have seen the
achieved a respectable measure of unity *** at most *** levels of the party .
                 with a debt which is , *** at the most *** charitable interpretation , expensive .
\end{verbatim}}

There are many more of these than of \q{at (the (very)?)? least/most
N}. A few of them may actually be instances of this pattern,
e.g. \q{There 's at least an hour or so of sunshine} is pretty similar
to \q{There 's at least one hour or so of sunshine}, so maybe (maybe) \q{an}
here is a number. Of the others, it is notable in
a good number of cases that \q{at the very least} is wrapped in a pair of
parenthetcal commas; and that examples involving \q{most} tend to have
multiple readings -- one where \q{at (the (very)?)? most} is a
specifier, and one where the whole thing is a PP (\q{at most levels
  of the party}, \q{at the most charitable interpretation}). It's
hard to see how you would disambiguate this unless the surrcounding
syntactic context did it for you (which it will in these two examples,
so maybe everything will just come out in the wash).

I think the big question here is how strong is the connection between
\q{at least N} and just plain \q{at least}. I don't see much wrong
with the interpretation above for \q{at least N}, and while there may
or may not be neat ways of computing whether or not \q{At least six dogs won
  prizes} contradicts \q{At most three animals won prizes}, I think
that saying that there was a set of dogs of cardinality $\ge$ 6 is
about right. But obviously this won't work for things like \q{This
  makes the anthropologist at the least an uncomfortable
  associate}. So either the \q{at the least} has two different
meanings, one when it combines with a number to make a complex
specifier and one when it combines with something else, which would be
annoying; or just saying that it's about the cardinality of a set
won't work.

\bibliography{refs}
\end{document}